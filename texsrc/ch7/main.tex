\section{関手圏}
\begin{Prop}
小圏$\mathscr{A}$と局所小圏$\mathscr{B}$に対して, 次の集合の組$(S_{\rm obj},S_{\rm mor})$は圏をなす.
\begin{align*}
S_{\rm obj}&=\{F:\mathscr{A}\rightarrow\mathscr{B}\}\\
S_{\rm mor}&=\{\alpha:F_1\rightarrow F_2\mid F_1\in S_{\rm obj},F_2\in S_{\rm obj}\}
\end{align*}
\end{Prop}
\begin{Def}
小圏$\mathscr{A}$と局所小圏$\mathscr{B}$に対する次の集合の組$(S_{\rm obj},S_{\rm mor})$からなる圏を{\bf 小圏$\mathscr{A}$から局所小圏$\mathscr{B}$への関手圏 functor category}という.
\begin{align*}
S_{\rm obj}&=\{F:\mathscr{A}\rightarrow\mathscr{B}\}\\
S_{\rm mor}&=\{\alpha:F_1\rightarrow F_2\mid F_1\in S_{\rm obj},F_2\in S_{\rm obj}\}
\end{align*}

\end{Def}
\begin{Notation}
小圏$\mathscr{A}$から局所小圏$\mathscr{B}$への関手圏を$[\mathscr{A},\mathscr{B}]$
で表す\end{Notation}
\begin{Def}
関手圏における同型射を{\bf 自然同型 
 natural isomorphism}という.
\end{Def}
\begin{comment}
\begin{Prop}
自然変換...が自然同型であること,

...が同型者であることは同値である.
\end{Prop}
\end{comment}