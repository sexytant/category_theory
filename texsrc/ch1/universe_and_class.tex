\section{宇宙}
\begin{comment}
{\bf 内包原理}とは「集合$S$の元$x$に対してtrue か falseを返す関数$\varphi:S\rightarrow\{\mathrm{true},\mathrm{false}\}$が与えられたとき, 新たな集合
$
\{x\in S\mid\varphi(x)=\mathrm{true}\}
$
を構成できる」という集合が満たすべき性質のことをいう. 

ここまで, 集合を「ものの集まり」と素朴に定義してきたが, 次のような集合$R$を内包原理に基づいて構成しようとすると矛盾が生じる.\footnote{ラッセルのパラドックスと呼ばれる}
\[
R=\{X\mid X\notin X\}
\]
もし$R\in R$であると仮定すると, $R$の定義により$R\notin R$となる.
他方, $R\notin R$と仮定すると, $R$の定義より$R\in R$となってしまう.

このような矛盾が発生しないようにするため, {\bf 宇宙universe}という概念を導入する.
\end{comment}
\begin{Def}
以下の性質を満たす集合$U$を{\bf 宇宙universe}という.
\begin{enumerate}
\item $X\in Y\land Y\in U\Rightarrow X\in U$
\item $X\in U\land Y\in U\Rightarrow\{X,Y\}\in U$
\item $X\in U\Rightarrow \mathcal{P}X\in U\land \cup X\in U$
\item $\mathbb{N}\in U$
\item $f:A\rightarrow B$が全射で, $A\in U\land B\subset U\Rightarrow B \in U$
\end{enumerate}
\end{Def}
\begin{Def}
宇宙の元を{\bf 小集合 small set}という.
\end{Def}
\begin{comment}
\begin{Def}
$a$と$b$が小集合のとき関数$f:a\rightarrow b$を{\bf 小関数 small function}という.
\end{Def}

\begin{caution}
「宇宙」は小集合の全体である.
\end{caution}
\begin{Prop}
「宇宙」は小集合ではない
\end{Prop}
\section{類}
\begin{Def}
「宇宙」の部分集合を{\bf 類 class} という.
\end{Def}
\begin{Prop}
「宇宙」は類である.
\end{Prop}

\begin{Def}
小集合でない類を{\bf 真類 proper class}という.
\end{Def}
\begin{Prop}
「宇宙」は真類である.
\end{Prop}
\end{comment}