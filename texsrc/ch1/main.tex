\section{集合}
{\bf 集合 set}\index{しゅうごう@集合}
とは, ひとまず素朴に「1つ以上のものの集まり」と定義される.
\footnote{本書は{\bf ZFC公理系}のもとで議論を進めるが, これについては深く述べない.} 
集合を構成する「もの」を{\bf 元 element}\index{げん@元}という.
\begin{Notation}
$a$が集合$S$の元であることを$a\in S$で表す.
\end{Notation}
\begin{Notation}
集合は各元をカンマで区切り$\{\}$で囲むことで表される. 
\end{Notation}
\begin{Notation}
とある条件を満たす$x$全体からなる集合は$\{x\mid x\text{についての条件}\}$を用いて表す.
\end{Notation}
\begin{example}$\{1,2,3\}$や$\{a,b,c\}$は集合である. 
\end{example}
\begin{example}$\{1\}$や$\{a\}$は集合である. 
\end{example}
\begin{example}
$\{2a\mid a\in\mathbb{N}\}$や$\{a^3\mid a\in\mathbb{N}\}$は集合である.
\end{example}

\begin{example}
自然数全体$\mathbb{N}$, 整数全体$\mathbb{Z}$, 実数全体$\mathbb{R}$, 複素数全体$\mathbb{C}$は集合である.
\end{example}
\begin{caution}
ここでは, 集合の元もまた集合であるとする.
\footnote{このような集合を{\bf 遺伝的集合}というが, ここでは単に集合と呼ぶ.} 
\begin{itemize}
\item 
$1$を$\{1\}$と同一視する.
\item 
$a$を$\{a\}$と同一視する.
\item
$\{a,b,c\}$を$\{\{a\},\{b\},\{c\}\}$と同一視する.
\item 
$\{a,b,\{c,d,e\}\}$を$\{\{a\},\{b\},\{c,d,e\}\}$と同一視する.
\end{itemize}
\end{caution}
\subsection{集合の合併}
\begin{Def}
集合$S$に対して
$\{x\in X\mid X\in S\}$
を{\bf 集合$S$の合併 union}\index{がっぺい@合併}という.
\end{Def}
\begin{Notation}
集合$S$の合併を$\cup S$で表す.
\end{Notation}
\begin{example}
$S=\{\{a,b\},\{c,d,e\}\}$のとき$\cup S=\{a,b,c,d,e\}$となる.
\end{example}
\begin{example}
$S=\{a,b,\{c,d,e\}\}$のとき, $S$と$\{\{a\},\{b\},\{c,d,e\}\}$を同一視するので,
$\cup S=\{a,b,c,d,e\}$となる.
\end{example}
\subsection{空集合}
元が存在しない集合$\{\}$の存在は認められる.
\footnote{{\bf ZF公理系}において, {\bf 外延性公理}と{\bf 内包公理}から空集合の存在が導かれることが知られているが, 本書では扱わない.}
\begin{Def}
元が存在しない集合$\{\}$を{\bf 空集合 emptyset}\index{くうしゅうごう@空集合}という.
\end{Def}
\begin{Notation}
空集合を$\emptyset$で表す.
\end{Notation}
\subsection{部分集合}
\begin{Def}
集合$A$の任意の元が集合$B$の元であるとき{\bf 集合$A$は集合$B$の部分集合 subset\index{ぶぶんしゅうごう@部分集合}である}という.  
\end{Def}
\begin{Notation}
集合$A$が集合$B$の部分集合であることを$A\subset B$で表す.
\end{Notation}
\subsection{冪集合}
\begin{Def}
集合$S$に対する$\{A\mid A\subset S\}$
を
{\bf 集合$S$の
冪集合 power set\index{べきしゅうごう@冪集合}}
という.
\end{Def}
\begin{Notation}
集合$S$の冪集合を$\mathcal{P}S$で表す.
\end{Notation}
\begin{example}
$S=\{a,b,c\}$のとき$\mathcal{P} S=\{\{a\},\{b\},\{c\},\{a,b\},\{a,c\},\{b,c\},\{a,b,c\}\}$となる.
\end{example}

\subsection{集合の直積}
\begin{Def}
集合$A,B$に対する$\{(a,b)\mid a\in A, b\in B\}$を
{\bf 集合$A$と集合$B$の
直積 direct product\index{ちょくせき@直積}}という.
\end{Def}
\begin{Notation}
集合$A$と集合$B$の
直積を$A\times B$で表す.
\end{Notation}
\begin{comment}
実は, 集合を「ものの集まり」と素朴に定義することは, 矛盾を孕んでいる. この矛盾を避けるための議論はのちに行う. 
\end{comment}
\section{二項関係・写像}
\subsection{二項関係}
\begin{Def}
集合$A,B$の直積の部分集合を, 
{\bf 集合$A$と集合$B$の
二項関係 binary relation}\index{にこうかんけい@二項関係}
という.
\end{Def}

\subsection{順序関係}
\begin{Def}
集合$X$とそれ自身$X$の二項関係$\mathrm{R}$が次を満たすとき$\mathrm{R}$を
{\bf 順序関係 order\index{じゅんじょかんけい@順序関係}}
という.
\begin{enumerate}
\item 任意の$x\in X$について$(x,x)\in \mathrm{R}$である.
\item 任意の$x\in X,y\in X$について$(x,y)\in \mathrm{R}$かつ$(y,x)\in \mathrm{R}$ならば$x=y$が成り立つ.
\item 任意の$x\in X,y\in X,z\in X$について$(x,y)\in \mathrm{R}$かつ $(y,z)\in \mathrm{R}$ならば
$(x,z)\in \mathrm{R}$が成り立つ.
\end{enumerate}
\end{Def}
\begin{Def}
順序関係をもつ集合を
{\bf 順序集合 orderd set\index{じゅんじょしゅうごう@順序集合}}
という.
\end{Def}
\begin{Notation}
順序集合の元$x,y$に順序関係があるとき, $x\preceq y$で表す.
\end{Notation}
\subsection{写像}
\begin{Def}
集合$A,B$の二項関係$f$について,
任意の$a\in A$に対して, とある$b\in B$が一意に存在して$(a,b)\in f$となるとき,
$f$を
{\bf 集合$A$から集合$B$への
写像map\index{しゃぞう@写像}}
という.
\end{Def}
\begin{Notation}
集合$A$から集合$B$への写像$f$を$f:A\rightarrow B$で表す.
\end{Notation}
\begin{Notation}
写像$f:A\rightarrow B$に関して, $b\in B$が$a\in A$に対応することを
\[
b=f(a)
\]
で表す.
\end{Notation}
\begin{Def}
写像$f:A\rightarrow B$が,
任意の$a_1,a_2\in A$について
\[
a_1\neq a_2\Rightarrow f(a_1)\neq f(a_2)
\]
を満たすとき
{\bf 写像$f:A\rightarrow B$は
単射 injection\index{たんしゃ@単射}
である}という.
\end{Def}
\begin{Def}
写像$f:A\rightarrow B$に関して,
任意の$b\in B$に対して
とある$a\in A$が存在して
\[
b=f(a)
\]
であるとき
{\bf 写像$f:A\rightarrow B$は
全射surjection\index{ぜんしゃ@全射} 
である}という.
\end{Def}
\begin{Def}
写像$f:A\rightarrow B$が, 単射であり, かつ全射であるとき
{\bf 写像$f:A\rightarrow B$は
全単射 bijection\index{ぜんたんしゃ@全単射}
である}という.
\end{Def}

\begin{Def}
写像$f:A\rightarrow B$に対して,写像$g:B\rightarrow A$が存在して, 任意の$a\in A$に対して
\[
f(a)=b \Leftrightarrow g(b)=a
\]
が成り立つとき, 
{\bf 写像$g:B\rightarrow A$は写像$f:A\rightarrow B$の
逆写像 inverse mapping\index{ぎゃくしゃぞう@逆写像}
である}という.
\end{Def}
\begin{Notation}
写像$f$の逆写像を$f^{-1}$で表す.
\end{Notation}

\subsection{合成と結合律}
\begin{Def}
二項関係$F\subset A\times B$と$G\subset B\times C$に対する
\[
\{(a,c)\in A\times C\mid \text{とある}b\in B\text{が存在して}(a,b)\in F \text{かつ} (b,c)\in G\text{である}\}
\]
を
{\bf 二項関係$F\subset A\times B$と$G\subset B\times C$の
合成 composition\index{ごうせい@合成}
}という.
\end{Def}
\begin{Notation}
二項関係$F\subset A\times B$と$G\subset B\times C$の合成を$G\circ F$で表す.
\end{Notation}
\begin{Prop}
二項関係$F\subset A\times B ,G\subset B\times C, H\subset C\times D$について
\[
(H\circ G)\circ F=H\circ (G\circ F)
\]
が成り立つ.
\end{Prop}

このことを二項関係が
{\bf 結合律 associative law\index{けつごうりつ@結合律}}
を満たすという.
\begin{Prop}
写像$f:A\rightarrow B$と$g:B\rightarrow C$に対して,
これらの合成を定義することができる.
\end{Prop}

\begin{Def}
写像$f:A\rightarrow B$と$g:B\rightarrow C$を合成して得られる写像を
{\bf 写像$f:A\rightarrow B$と$g:B\rightarrow C$の
合成写像 composition mapping\index{ごうせいしゃぞう@合成写像}}
という.
\end{Def}
\begin{Notation}
写像$f:A\rightarrow B$と$g:B\rightarrow C$の合成写像を$g\circ f$で表す.
\end{Notation}
\begin{Prop}
写像$f:A\rightarrow B, g:B\rightarrow C,
h:C\rightarrow D$について
\[
(h\circ g)\circ f=h\circ(g\circ f)
\]
が成り立つ.
\end{Prop}

このことを写像が結合律を満たすという.
\section{宇宙}
\begin{comment}
{\bf 内包原理}とは「集合$S$の元$x$に対してtrue か falseを返す関数$\varphi:S\rightarrow\{\mathrm{true},\mathrm{false}\}$が与えられたとき, 新たな集合
$
\{x\in S\mid\varphi(x)=\mathrm{true}\}
$
を構成できる」という集合が満たすべき性質のことをいう. 

ここまで, 集合を「ものの集まり」と素朴に定義してきたが, 次のような集合$R$を内包原理に基づいて構成しようとすると矛盾が生じる.\footnote{ラッセルのパラドックスと呼ばれる}
\[
R=\{X\mid X\notin X\}
\]
もし$R\in R$であると仮定すると, $R$の定義により$R\notin R$となる.
他方, $R\notin R$と仮定すると, $R$の定義より$R\in R$となってしまう.

このような矛盾が発生しないようにするため, {\bf 宇宙universe}という概念を導入する.
\end{comment}
\begin{Def}
以下の性質を満たす集合$U$を
{\bf 宇宙universe\index{うちゅう@宇宙}}
という.
\begin{enumerate}
\item $X\in Y\land Y\in U\Rightarrow X\in U$
\item $X\in U\land Y\in U\Rightarrow\{X,Y\}\in U$
\item $X\in U\Rightarrow \mathcal{P}X\in U\land \cup X\in U$
\item $\mathbb{N}\in U$
\item $f:A\rightarrow B$が全射で, $A\in U\land B\subset U\Rightarrow B \in U$
\end{enumerate}
\end{Def}
\begin{Def}
宇宙の元を
{\bf 小集合 small set\index{しょうしゅうごう@小集合}}
という.
\end{Def}
\begin{comment}
\begin{Def}
$a$と$b$が小集合のとき関数$f:a\rightarrow b$を{\bf 小関数 small function}という.
\end{Def}

\begin{caution}
「宇宙」は小集合の全体である.
\end{caution}
\begin{Prop}
「宇宙」は小集合ではない
\end{Prop}
\section{類}
\begin{Def}
「宇宙」の部分集合を{\bf 類 class} という.
\end{Def}
\begin{Prop}
「宇宙」は類である.
\end{Prop}

\begin{Def}
小集合でない類を{\bf 真類 proper class}という.
\end{Def}
\begin{Prop}
「宇宙」は真類である.
\end{Prop}
\end{comment}

\begin{comment}
\section{命題の証明}
\begin{color}{red}このセクションはレビュー対象外\end{color}

命題1.26の証明
\begin{proof}
\end{proof}
命題1.27の証明
\begin{proof}
\end{proof}
命題1.30の証明
\begin{proof}
\end{proof}
\end{comment}