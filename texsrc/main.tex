\documentclass[dvipdfmx]{jsbook}

% Language setting
% Replace `english' with e.g. `spanish' to change the document language

% Set page size and margins
% Replace `letterpaper' with `a4paper' for UK/EU standard size
\usepackage[letterpaper,top=2cm,bottom=2cm,left=3cm,right=3cm,marginparwidth=1.75cm]{geometry}

% Useful packages
\usepackage{amsmath}
\usepackage{amssymb}
\usepackage{graphicx}
\usepackage[colorlinks=true, allcolors=blue]{hyperref}
\usepackage{tikz-cd}
\usepackage{amsthm}
\usepackage{mathrsfs}
\usepackage{comment}
\usepackage{hyperref}
\usepackage{pxjahyper}
\usepackage{bm}
\theoremstyle{plain}

\newtheorem{thm}{定理}[chapter]
\newtheorem{Def}[thm]{定義}
\newtheorem{Notation}[thm]{記法}
\newtheorem{Prop}[thm]{命題}
\newtheorem{caution}[thm]{注意}
\newtheorem{example}[thm]{実例}
\renewcommand{\proofname}{\textbf{証明}}
\usepackage{makeidx}
\usepackage{comment}
\usepackage{listings}
\lstset{
	%プログラム言語(複数の言語に対応,C,C++も可)
 	language = haskell,
 	%背景色と透過度
 	backgroundcolor={\color[gray]{.90}},
 	%枠外に行った時の自動改行
 	breaklines = true,
 	%自動開業後のインデント量(デフォルトでは20[pt])	
 	breakindent = 10pt,
 	%標準の書体
 	basicstyle = \ttfamily\normalsize,
 	%basicstyle = {\small}
 	%コメントの書体
 	commentstyle = {\itshape \color[cmyk]{1,0.4,1,0}},
 	%関数名等の色の設定
 	classoffset = 0,
 	%キーワード(int, ifなど)の書体
 	keywordstyle = {\bfseries \color[cmyk]{0,1,0,0}},
 	%""で囲まれたなどの"文字"の書体
 	stringstyle = {\ttfamily \color[rgb]{0,0,1}},
 	%枠 "t"は上に線を記載, "T"は上に二重線を記載
	%他オプション:leftline,topline,bottomline,lines,single,shadowbox
 	frame = TBrl,
 	%frameまでの間隔(行番号とプログラムの間)
 	framesep = 5pt,
 	%行番号の位置
 	numbers = left,
	%行番号の間隔
 	stepnumber = 1,
	%右マージン
 	%xrightmargin=0zw,
 	%左マージン
	%xleftmargin=3zw,
	%行番号の書体
 	numberstyle = \tiny,
	%タブの大きさ
 	tabsize = 4,
 	%キャプションの場所("tb"ならば上下両方に記載)
 	captionpos = t
}

\makeindex
\title{圏論原論I}
\author{Sexytant}
\begin{document}
\setcounter{tocdepth}{2}
\maketitle
\tableofcontents
\newpage
\part{準備}
\chapter{集合}
\section{関手}
\begin{Def}
写像$f:\mathrm{Obj}(\mathscr{A})\rightarrow\mathrm{Obj}(\mathscr{B})$を
{\bf 圏$\mathscr{A}$から圏$\mathscr{B}$への
対象関数 function on objects\index{たいしょうかんすう@対象関数}
}という.
\end{Def}
\begin{Notation}
圏$\mathscr{A}$から圏$\mathscr{B}$への対象関数を$F_{\rm obj}(\mathscr{A},\mathscr{B})$で表す.
\end{Notation}
\begin{Def}
圏$\mathscr{A}$の射全体の集合\[\mathrm{Mor}(\mathscr{A})=\{f_{\mathscr{A}}:A_1\rightarrow A_2\mid A_1\in\mathrm{Obj}(\mathscr{A}),A_2\in\mathrm{Obj}(\mathscr{A})\}\]
から, 対象関数$F_{obj}(\mathscr{A},\mathscr{B})(\cdot)$を用いて定まる, 圏$\mathscr{B}$の射の集合
\[
\mathrm{Mor}(\mathscr{A}\rightarrow\mathscr{B})=\{f_{\mathscr{B}}:F_{\rm obj}(\mathscr{A},\mathscr{B})(A_1)\rightarrow
F_{\rm obj}(\mathscr{A},\mathscr{B})(A_2)\mid A_1\in\mathrm{Obj}(\mathscr{A}),A_2\in\mathrm{Obj}(\mathscr{A})\}
\]
への写像$F_{\rm mor}(\mathscr{A}\rightarrow\mathscr{B}):\mathrm{Mor}(\mathscr{A})\rightarrow\mathrm{Mor}(\mathscr{A}\rightarrow\mathscr{B})$
が次の条件を満たすとき, これを
{\bf 圏$\mathscr{A}$から圏$\mathscr{B}$への
射関数 function on morphisms\index{しゃかんすう@射関数}
}という.
\begin{enumerate}
\item 任意の$A\in\mathrm{Obj}(\mathscr{A})$に対して
\[
F_{\rm mor}(\mathscr{A}\rightarrow\mathscr{B})
(1_A)=1_{F_{\rm mor}(\mathscr{A}\rightarrow\mathscr{B})(A)}\]
が成り立つ.
\item 任意の$(A_1,A_2,A_3)\in \mathrm{Obj}(\mathscr{A})^3$
と$f_{1,2}:A_1\rightarrow A_2, f_{2,3}:A_2\rightarrow A_3$に対して
\[
F_{\mathrm{Mor}}(\mathscr{A}\rightarrow\mathscr{B})(f_{2,3}\circ f_{1,2})
=F_{\mathrm{Mor}}(f_{2,3})\circ F_{\mathrm{Mod}}(f_{1,2}
\]
が成り立つ.
\end{enumerate}
\end{Def}
\begin{Notation}
圏$\mathscr{A}$から圏$\mathscr{B}$への射関数を
$F_{\rm mor}(\mathscr{A}\rightarrow\mathscr{B})$で表す.
\end{Notation}
\begin{Def}
対象関数$F_{\rm obj}(\mathscr{A},\mathscr{B})$と
射関数$F_{\rm mor}(\mathscr{A}\rightarrow\mathscr{B})$の組を{\bf 圏$\mathscr{A}$から圏$\mathscr{B}$への
関手 functor\index{かんしゅ@関手}
}という.
\end{Def}
\begin{Notation}
圏$\mathscr{A}$から圏$\mathscr{B}$への関手$F$を
$F:\mathscr{A}\rightarrow\mathscr{B}$で表す.
\end{Notation}
\begin{Def}
圏$\mathscr{C}$から圏$\mathscr{C}$への関手を{\bf 圏$\mathscr{C}$に関する
恒等関手 identity functor\index{こうとうかんしゅ@恒等関手}
}という.
\end{Def}
\begin{Notation}
圏$\mathscr{C}$に関する恒等関手を$\mathrm{Id}(\mathscr{C})$で表す.
\end{Notation}
\begin{Def}
関手$F:\mathscr{A}\rightarrow\mathscr{B},G:\mathscr{B}\rightarrow\mathscr{C}$に対して
\[F_{\rm obj}(\mathscr{B},\mathscr{C})\circ F_{\rm obj}(\mathscr{A},\mathscr{B}\]
を対象関数とし.
\[F_{\rm mor}(\mathscr{B}\rightarrow\mathscr{C})\circ F_{\rm mor}(\mathscr{A}\rightarrow\mathscr{A})\]
を射関数とする関手
を{\bf 関手$F:\mathscr{A}\rightarrow\mathscr{B},G:\mathscr{B}\rightarrow\mathscr{C}$の合成 composition}という.
\end{Def}
\begin{Notation}
関手$F:\mathscr{A}\rightarrow\mathscr{B},G:\mathscr{B}\rightarrow\mathscr{C}$の合成を$G\circ F$で表す.
\end{Notation}
\begin{comment}
\begin{example}
順序を保存する写像
\end{example}
\end{comment}
\begin{comment}
\begin{example}
圏と見做した順序集合間の簡単な関手の例
\end{example}
\end{comment}
\begin{comment}
*************************
\begin{example}
$n$次ホモロジー関手
\end{example}
***************************
\end{comment}

\subsection{反変関手}
\begin{Def}
関手$F:\mathscr{A}^{\mathrm{op}}\rightarrow\mathscr{B}$
を{\bf 圏$\mathscr{A}$から圏$\mathscr{B}$への
反変関手 contravariant functor\index{はんへんかんしゅ@反変関手}
}という.
\end{Def}
\subsection{定数関手}
\begin{Def}
任意の$A\in\mathrm{Obj}(\mathscr{A})$を唯一の$B_0\in\mathrm{Obj}(\mathscr{B})$に写し,
任意の射$f\in\mathrm{Mor}(\mathscr{A})$を恒等射$1_{B_0}\in\mathrm{Mor}(\mathscr{B})$に写す関手を
{\bf 圏$\mathscr{A}$から圏$\mathscr{B}$への
定数関手 constant functor\index{ていすうかんしゅ@定数関手}
}という
\end{Def}

\subsection{忠実関手と充満関手}
\begin{Def}
関手$F:\mathscr{A}\rightarrow\mathscr{B}$に関して,
写像
\[f:\{(A_1,A_2)\mid A_1\in\mathrm{Obj}(\mathscr{A}),A_2\in\mathrm{Obj}(\mathscr{A})\}\rightarrow\{(F(A_1),F(A_2))\mid A_1\in\mathrm{Obj}(\mathscr{A}),A_2\in\mathrm{Obj}(\mathscr{A})\}\]
が単射となっているとき,
{\bf 関手$F$は
忠実 faithful\index{ちゅうじつ@忠実}
である}という.
\end{Def}
\begin{Def}
関手$F:\mathscr{A}\rightarrow\mathscr{B}$に関して,
写像
\[f:\{(A_1,A_2)\mid A_1\in\mathrm{Obj}(\mathscr{A}),A_2\in\mathrm{Obj}(\mathscr{A})\}\rightarrow\{(F(A_1),F(A_2))\mid A_1\in\mathrm{Obj}(\mathscr{A}),A_2\in\mathrm{Obj}(\mathscr{A})\}\]
が全射となっているとき,
{\bf 関手$F$は
充満 full\index{じゅうまん@充満} 
である}という.
\end{Def}

\begin{Def}
関手$F$が忠実かつ充満であるとき
{\bf 関手$F$は
充満忠実 full and faithful\index{じゅうまんちゅうじつ@充満忠実}
である}という.
\end{Def}

\begin{Def}
圏$\mathscr{A}$が圏$\mathscr{A}$の部分圏であり, 関手$F:\mathscr{A}\rightarrow\mathscr{B}$が充満であるとき,
{\bf 圏$\mathscr{A}$は圏 $\mathscr{B}$の充満部分圏である}という.
\end{Def}

\begin{comment}
\begin{example}
...充満忠実である.
\end{example}
\begin{example}
...忠実だが充満でない
\end{example}
\begin{example}
充満だが忠実でない
\end{example}
\begin{example}
複素数...

...

...忠実だが充満でない. (例1.30)
\end{example}
\end{comment}
\begin{comment}
\subsection{埋め込み関手}
\subsection{忘却関手}
\end{comment}
\begin{comment}
\section{Hask上の関手}
\subsection{List関手(\S 5に挿入)}



\begin{Prop}
Haskellにおける型構築子\verb|[]|は, HaskからHaskへの対象関数である.
\end{Prop}
\begin{proof}
型構築子\verb|[]|は, 任意の型\verb|A|に対して型\verb|[A]|を対応させるので, 命題が成り立つ.
\end{proof}

型\verb|A|と型\verb|B|および関数\verb|f::A->B|が与えられとき\verb|map f::[A]->[B]|が決定される.
\begin{Prop}
\verb|map|はHaskからHaskへの射関数である.
\end{Prop}
\begin{Prop}
型構築子\verb|[]|と\verb|map|がHaskからHaskへの関手をなす.
\end{Prop}
\begin{proof}
命題...と命題...より明らか.
\end{proof}
型構築子\verb|[]|を{\bf List関手}\index{りすとかんしゅ@List関手}という.
\subsubsection{Maybe関手}
Haskellにおける型構築子\verb|Maybe|は...

...

型構築子\verb|Maybe|は{\bf Maybe関手}\index{めいびーかんしゅ@Maybe関手}
と呼ばれる.
\subsubsection{Tree関手}
一般に木構造を生成する型構築子は関手にできる. これを{\bf Tree関手}と呼ぶ.

\lstinputlisting{../hssrc/Tree.hs}
\section{2変数の関手}
{\bf Hom関手}
\lstinputlisting{../hssrc/homfunctors.hs}
\section{型クラスとHaskの部分圏}
{\bf ソート関手}
\lstinputlisting{../hssrc/sort.hs}
\end{comment}



\chapter{代数系}
\section{関手}
\begin{Def}
写像$f:\mathrm{Obj}(\mathscr{A})\rightarrow\mathrm{Obj}(\mathscr{B})$を
{\bf 圏$\mathscr{A}$から圏$\mathscr{B}$への
対象関数 function on objects\index{たいしょうかんすう@対象関数}
}という.
\end{Def}
\begin{Notation}
圏$\mathscr{A}$から圏$\mathscr{B}$への対象関数を$F_{\rm obj}(\mathscr{A},\mathscr{B})$で表す.
\end{Notation}
\begin{Def}
圏$\mathscr{A}$の射全体の集合\[\mathrm{Mor}(\mathscr{A})=\{f_{\mathscr{A}}:A_1\rightarrow A_2\mid A_1\in\mathrm{Obj}(\mathscr{A}),A_2\in\mathrm{Obj}(\mathscr{A})\}\]
から, 対象関数$F_{obj}(\mathscr{A},\mathscr{B})(\cdot)$を用いて定まる, 圏$\mathscr{B}$の射の集合
\[
\mathrm{Mor}(\mathscr{A}\rightarrow\mathscr{B})=\{f_{\mathscr{B}}:F_{\rm obj}(\mathscr{A},\mathscr{B})(A_1)\rightarrow
F_{\rm obj}(\mathscr{A},\mathscr{B})(A_2)\mid A_1\in\mathrm{Obj}(\mathscr{A}),A_2\in\mathrm{Obj}(\mathscr{A})\}
\]
への写像$F_{\rm mor}(\mathscr{A}\rightarrow\mathscr{B}):\mathrm{Mor}(\mathscr{A})\rightarrow\mathrm{Mor}(\mathscr{A}\rightarrow\mathscr{B})$
が次の条件を満たすとき, これを
{\bf 圏$\mathscr{A}$から圏$\mathscr{B}$への
射関数 function on morphisms\index{しゃかんすう@射関数}
}という.
\begin{enumerate}
\item 任意の$A\in\mathrm{Obj}(\mathscr{A})$に対して
\[
F_{\rm mor}(\mathscr{A}\rightarrow\mathscr{B})
(1_A)=1_{F_{\rm mor}(\mathscr{A}\rightarrow\mathscr{B})(A)}\]
が成り立つ.
\item 任意の$(A_1,A_2,A_3)\in \mathrm{Obj}(\mathscr{A})^3$
と$f_{1,2}:A_1\rightarrow A_2, f_{2,3}:A_2\rightarrow A_3$に対して
\[
F_{\mathrm{Mor}}(\mathscr{A}\rightarrow\mathscr{B})(f_{2,3}\circ f_{1,2})
=F_{\mathrm{Mor}}(f_{2,3})\circ F_{\mathrm{Mod}}(f_{1,2}
\]
が成り立つ.
\end{enumerate}
\end{Def}
\begin{Notation}
圏$\mathscr{A}$から圏$\mathscr{B}$への射関数を
$F_{\rm mor}(\mathscr{A}\rightarrow\mathscr{B})$で表す.
\end{Notation}
\begin{Def}
対象関数$F_{\rm obj}(\mathscr{A},\mathscr{B})$と
射関数$F_{\rm mor}(\mathscr{A}\rightarrow\mathscr{B})$の組を{\bf 圏$\mathscr{A}$から圏$\mathscr{B}$への
関手 functor\index{かんしゅ@関手}
}という.
\end{Def}
\begin{Notation}
圏$\mathscr{A}$から圏$\mathscr{B}$への関手$F$を
$F:\mathscr{A}\rightarrow\mathscr{B}$で表す.
\end{Notation}
\begin{Def}
圏$\mathscr{C}$から圏$\mathscr{C}$への関手を{\bf 圏$\mathscr{C}$に関する
恒等関手 identity functor\index{こうとうかんしゅ@恒等関手}
}という.
\end{Def}
\begin{Notation}
圏$\mathscr{C}$に関する恒等関手を$\mathrm{Id}(\mathscr{C})$で表す.
\end{Notation}
\begin{Def}
関手$F:\mathscr{A}\rightarrow\mathscr{B},G:\mathscr{B}\rightarrow\mathscr{C}$に対して
\[F_{\rm obj}(\mathscr{B},\mathscr{C})\circ F_{\rm obj}(\mathscr{A},\mathscr{B}\]
を対象関数とし.
\[F_{\rm mor}(\mathscr{B}\rightarrow\mathscr{C})\circ F_{\rm mor}(\mathscr{A}\rightarrow\mathscr{A})\]
を射関数とする関手
を{\bf 関手$F:\mathscr{A}\rightarrow\mathscr{B},G:\mathscr{B}\rightarrow\mathscr{C}$の合成 composition}という.
\end{Def}
\begin{Notation}
関手$F:\mathscr{A}\rightarrow\mathscr{B},G:\mathscr{B}\rightarrow\mathscr{C}$の合成を$G\circ F$で表す.
\end{Notation}
\begin{comment}
\begin{example}
順序を保存する写像
\end{example}
\end{comment}
\begin{comment}
\begin{example}
圏と見做した順序集合間の簡単な関手の例
\end{example}
\end{comment}
\begin{comment}
*************************
\begin{example}
$n$次ホモロジー関手
\end{example}
***************************
\end{comment}

\subsection{反変関手}
\begin{Def}
関手$F:\mathscr{A}^{\mathrm{op}}\rightarrow\mathscr{B}$
を{\bf 圏$\mathscr{A}$から圏$\mathscr{B}$への
反変関手 contravariant functor\index{はんへんかんしゅ@反変関手}
}という.
\end{Def}
\subsection{定数関手}
\begin{Def}
任意の$A\in\mathrm{Obj}(\mathscr{A})$を唯一の$B_0\in\mathrm{Obj}(\mathscr{B})$に写し,
任意の射$f\in\mathrm{Mor}(\mathscr{A})$を恒等射$1_{B_0}\in\mathrm{Mor}(\mathscr{B})$に写す関手を
{\bf 圏$\mathscr{A}$から圏$\mathscr{B}$への
定数関手 constant functor\index{ていすうかんしゅ@定数関手}
}という
\end{Def}

\subsection{忠実関手と充満関手}
\begin{Def}
関手$F:\mathscr{A}\rightarrow\mathscr{B}$に関して,
写像
\[f:\{(A_1,A_2)\mid A_1\in\mathrm{Obj}(\mathscr{A}),A_2\in\mathrm{Obj}(\mathscr{A})\}\rightarrow\{(F(A_1),F(A_2))\mid A_1\in\mathrm{Obj}(\mathscr{A}),A_2\in\mathrm{Obj}(\mathscr{A})\}\]
が単射となっているとき,
{\bf 関手$F$は
忠実 faithful\index{ちゅうじつ@忠実}
である}という.
\end{Def}
\begin{Def}
関手$F:\mathscr{A}\rightarrow\mathscr{B}$に関して,
写像
\[f:\{(A_1,A_2)\mid A_1\in\mathrm{Obj}(\mathscr{A}),A_2\in\mathrm{Obj}(\mathscr{A})\}\rightarrow\{(F(A_1),F(A_2))\mid A_1\in\mathrm{Obj}(\mathscr{A}),A_2\in\mathrm{Obj}(\mathscr{A})\}\]
が全射となっているとき,
{\bf 関手$F$は
充満 full\index{じゅうまん@充満} 
である}という.
\end{Def}

\begin{Def}
関手$F$が忠実かつ充満であるとき
{\bf 関手$F$は
充満忠実 full and faithful\index{じゅうまんちゅうじつ@充満忠実}
である}という.
\end{Def}

\begin{Def}
圏$\mathscr{A}$が圏$\mathscr{A}$の部分圏であり, 関手$F:\mathscr{A}\rightarrow\mathscr{B}$が充満であるとき,
{\bf 圏$\mathscr{A}$は圏 $\mathscr{B}$の充満部分圏である}という.
\end{Def}

\begin{comment}
\begin{example}
...充満忠実である.
\end{example}
\begin{example}
...忠実だが充満でない
\end{example}
\begin{example}
充満だが忠実でない
\end{example}
\begin{example}
複素数...

...

...忠実だが充満でない. (例1.30)
\end{example}
\end{comment}
\begin{comment}
\subsection{埋め込み関手}
\subsection{忘却関手}
\end{comment}
\begin{comment}
\section{Hask上の関手}
\subsection{List関手(\S 5に挿入)}



\begin{Prop}
Haskellにおける型構築子\verb|[]|は, HaskからHaskへの対象関数である.
\end{Prop}
\begin{proof}
型構築子\verb|[]|は, 任意の型\verb|A|に対して型\verb|[A]|を対応させるので, 命題が成り立つ.
\end{proof}

型\verb|A|と型\verb|B|および関数\verb|f::A->B|が与えられとき\verb|map f::[A]->[B]|が決定される.
\begin{Prop}
\verb|map|はHaskからHaskへの射関数である.
\end{Prop}
\begin{Prop}
型構築子\verb|[]|と\verb|map|がHaskからHaskへの関手をなす.
\end{Prop}
\begin{proof}
命題...と命題...より明らか.
\end{proof}
型構築子\verb|[]|を{\bf List関手}\index{りすとかんしゅ@List関手}という.
\subsubsection{Maybe関手}
Haskellにおける型構築子\verb|Maybe|は...

...

型構築子\verb|Maybe|は{\bf Maybe関手}\index{めいびーかんしゅ@Maybe関手}
と呼ばれる.
\subsubsection{Tree関手}
一般に木構造を生成する型構築子は関手にできる. これを{\bf Tree関手}と呼ぶ.

\lstinputlisting{../hssrc/Tree.hs}
\section{2変数の関手}
{\bf Hom関手}
\lstinputlisting{../hssrc/homfunctors.hs}
\section{型クラスとHaskの部分圏}
{\bf ソート関手}
\lstinputlisting{../hssrc/sort.hs}
\end{comment}



\chapter{Haskellの基礎}
\section{関手}
\begin{Def}
写像$f:\mathrm{Obj}(\mathscr{A})\rightarrow\mathrm{Obj}(\mathscr{B})$を
{\bf 圏$\mathscr{A}$から圏$\mathscr{B}$への
対象関数 function on objects\index{たいしょうかんすう@対象関数}
}という.
\end{Def}
\begin{Notation}
圏$\mathscr{A}$から圏$\mathscr{B}$への対象関数を$F_{\rm obj}(\mathscr{A},\mathscr{B})$で表す.
\end{Notation}
\begin{Def}
圏$\mathscr{A}$の射全体の集合\[\mathrm{Mor}(\mathscr{A})=\{f_{\mathscr{A}}:A_1\rightarrow A_2\mid A_1\in\mathrm{Obj}(\mathscr{A}),A_2\in\mathrm{Obj}(\mathscr{A})\}\]
から, 対象関数$F_{obj}(\mathscr{A},\mathscr{B})(\cdot)$を用いて定まる, 圏$\mathscr{B}$の射の集合
\[
\mathrm{Mor}(\mathscr{A}\rightarrow\mathscr{B})=\{f_{\mathscr{B}}:F_{\rm obj}(\mathscr{A},\mathscr{B})(A_1)\rightarrow
F_{\rm obj}(\mathscr{A},\mathscr{B})(A_2)\mid A_1\in\mathrm{Obj}(\mathscr{A}),A_2\in\mathrm{Obj}(\mathscr{A})\}
\]
への写像$F_{\rm mor}(\mathscr{A}\rightarrow\mathscr{B}):\mathrm{Mor}(\mathscr{A})\rightarrow\mathrm{Mor}(\mathscr{A}\rightarrow\mathscr{B})$
が次の条件を満たすとき, これを
{\bf 圏$\mathscr{A}$から圏$\mathscr{B}$への
射関数 function on morphisms\index{しゃかんすう@射関数}
}という.
\begin{enumerate}
\item 任意の$A\in\mathrm{Obj}(\mathscr{A})$に対して
\[
F_{\rm mor}(\mathscr{A}\rightarrow\mathscr{B})
(1_A)=1_{F_{\rm mor}(\mathscr{A}\rightarrow\mathscr{B})(A)}\]
が成り立つ.
\item 任意の$(A_1,A_2,A_3)\in \mathrm{Obj}(\mathscr{A})^3$
と$f_{1,2}:A_1\rightarrow A_2, f_{2,3}:A_2\rightarrow A_3$に対して
\[
F_{\mathrm{Mor}}(\mathscr{A}\rightarrow\mathscr{B})(f_{2,3}\circ f_{1,2})
=F_{\mathrm{Mor}}(f_{2,3})\circ F_{\mathrm{Mod}}(f_{1,2}
\]
が成り立つ.
\end{enumerate}
\end{Def}
\begin{Notation}
圏$\mathscr{A}$から圏$\mathscr{B}$への射関数を
$F_{\rm mor}(\mathscr{A}\rightarrow\mathscr{B})$で表す.
\end{Notation}
\begin{Def}
対象関数$F_{\rm obj}(\mathscr{A},\mathscr{B})$と
射関数$F_{\rm mor}(\mathscr{A}\rightarrow\mathscr{B})$の組を{\bf 圏$\mathscr{A}$から圏$\mathscr{B}$への
関手 functor\index{かんしゅ@関手}
}という.
\end{Def}
\begin{Notation}
圏$\mathscr{A}$から圏$\mathscr{B}$への関手$F$を
$F:\mathscr{A}\rightarrow\mathscr{B}$で表す.
\end{Notation}
\begin{Def}
圏$\mathscr{C}$から圏$\mathscr{C}$への関手を{\bf 圏$\mathscr{C}$に関する
恒等関手 identity functor\index{こうとうかんしゅ@恒等関手}
}という.
\end{Def}
\begin{Notation}
圏$\mathscr{C}$に関する恒等関手を$\mathrm{Id}(\mathscr{C})$で表す.
\end{Notation}
\begin{Def}
関手$F:\mathscr{A}\rightarrow\mathscr{B},G:\mathscr{B}\rightarrow\mathscr{C}$に対して
\[F_{\rm obj}(\mathscr{B},\mathscr{C})\circ F_{\rm obj}(\mathscr{A},\mathscr{B}\]
を対象関数とし.
\[F_{\rm mor}(\mathscr{B}\rightarrow\mathscr{C})\circ F_{\rm mor}(\mathscr{A}\rightarrow\mathscr{A})\]
を射関数とする関手
を{\bf 関手$F:\mathscr{A}\rightarrow\mathscr{B},G:\mathscr{B}\rightarrow\mathscr{C}$の合成 composition}という.
\end{Def}
\begin{Notation}
関手$F:\mathscr{A}\rightarrow\mathscr{B},G:\mathscr{B}\rightarrow\mathscr{C}$の合成を$G\circ F$で表す.
\end{Notation}
\begin{comment}
\begin{example}
順序を保存する写像
\end{example}
\end{comment}
\begin{comment}
\begin{example}
圏と見做した順序集合間の簡単な関手の例
\end{example}
\end{comment}
\begin{comment}
*************************
\begin{example}
$n$次ホモロジー関手
\end{example}
***************************
\end{comment}

\subsection{反変関手}
\begin{Def}
関手$F:\mathscr{A}^{\mathrm{op}}\rightarrow\mathscr{B}$
を{\bf 圏$\mathscr{A}$から圏$\mathscr{B}$への
反変関手 contravariant functor\index{はんへんかんしゅ@反変関手}
}という.
\end{Def}
\subsection{定数関手}
\begin{Def}
任意の$A\in\mathrm{Obj}(\mathscr{A})$を唯一の$B_0\in\mathrm{Obj}(\mathscr{B})$に写し,
任意の射$f\in\mathrm{Mor}(\mathscr{A})$を恒等射$1_{B_0}\in\mathrm{Mor}(\mathscr{B})$に写す関手を
{\bf 圏$\mathscr{A}$から圏$\mathscr{B}$への
定数関手 constant functor\index{ていすうかんしゅ@定数関手}
}という
\end{Def}

\subsection{忠実関手と充満関手}
\begin{Def}
関手$F:\mathscr{A}\rightarrow\mathscr{B}$に関して,
写像
\[f:\{(A_1,A_2)\mid A_1\in\mathrm{Obj}(\mathscr{A}),A_2\in\mathrm{Obj}(\mathscr{A})\}\rightarrow\{(F(A_1),F(A_2))\mid A_1\in\mathrm{Obj}(\mathscr{A}),A_2\in\mathrm{Obj}(\mathscr{A})\}\]
が単射となっているとき,
{\bf 関手$F$は
忠実 faithful\index{ちゅうじつ@忠実}
である}という.
\end{Def}
\begin{Def}
関手$F:\mathscr{A}\rightarrow\mathscr{B}$に関して,
写像
\[f:\{(A_1,A_2)\mid A_1\in\mathrm{Obj}(\mathscr{A}),A_2\in\mathrm{Obj}(\mathscr{A})\}\rightarrow\{(F(A_1),F(A_2))\mid A_1\in\mathrm{Obj}(\mathscr{A}),A_2\in\mathrm{Obj}(\mathscr{A})\}\]
が全射となっているとき,
{\bf 関手$F$は
充満 full\index{じゅうまん@充満} 
である}という.
\end{Def}

\begin{Def}
関手$F$が忠実かつ充満であるとき
{\bf 関手$F$は
充満忠実 full and faithful\index{じゅうまんちゅうじつ@充満忠実}
である}という.
\end{Def}

\begin{Def}
圏$\mathscr{A}$が圏$\mathscr{A}$の部分圏であり, 関手$F:\mathscr{A}\rightarrow\mathscr{B}$が充満であるとき,
{\bf 圏$\mathscr{A}$は圏 $\mathscr{B}$の充満部分圏である}という.
\end{Def}

\begin{comment}
\begin{example}
...充満忠実である.
\end{example}
\begin{example}
...忠実だが充満でない
\end{example}
\begin{example}
充満だが忠実でない
\end{example}
\begin{example}
複素数...

...

...忠実だが充満でない. (例1.30)
\end{example}
\end{comment}
\begin{comment}
\subsection{埋め込み関手}
\subsection{忘却関手}
\end{comment}
\begin{comment}
\section{Hask上の関手}
\subsection{List関手(\S 5に挿入)}



\begin{Prop}
Haskellにおける型構築子\verb|[]|は, HaskからHaskへの対象関数である.
\end{Prop}
\begin{proof}
型構築子\verb|[]|は, 任意の型\verb|A|に対して型\verb|[A]|を対応させるので, 命題が成り立つ.
\end{proof}

型\verb|A|と型\verb|B|および関数\verb|f::A->B|が与えられとき\verb|map f::[A]->[B]|が決定される.
\begin{Prop}
\verb|map|はHaskからHaskへの射関数である.
\end{Prop}
\begin{Prop}
型構築子\verb|[]|と\verb|map|がHaskからHaskへの関手をなす.
\end{Prop}
\begin{proof}
命題...と命題...より明らか.
\end{proof}
型構築子\verb|[]|を{\bf List関手}\index{りすとかんしゅ@List関手}という.
\subsubsection{Maybe関手}
Haskellにおける型構築子\verb|Maybe|は...

...

型構築子\verb|Maybe|は{\bf Maybe関手}\index{めいびーかんしゅ@Maybe関手}
と呼ばれる.
\subsubsection{Tree関手}
一般に木構造を生成する型構築子は関手にできる. これを{\bf Tree関手}と呼ぶ.

\lstinputlisting{../hssrc/Tree.hs}
\section{2変数の関手}
{\bf Hom関手}
\lstinputlisting{../hssrc/homfunctors.hs}
\section{型クラスとHaskの部分圏}
{\bf ソート関手}
\lstinputlisting{../hssrc/sort.hs}
\end{comment}



\part{圏論の諸概念}
\chapter{圏}
\section{関手}
\begin{Def}
写像$f:\mathrm{Obj}(\mathscr{A})\rightarrow\mathrm{Obj}(\mathscr{B})$を
{\bf 圏$\mathscr{A}$から圏$\mathscr{B}$への
対象関数 function on objects\index{たいしょうかんすう@対象関数}
}という.
\end{Def}
\begin{Notation}
圏$\mathscr{A}$から圏$\mathscr{B}$への対象関数を$F_{\rm obj}(\mathscr{A},\mathscr{B})$で表す.
\end{Notation}
\begin{Def}
圏$\mathscr{A}$の射全体の集合\[\mathrm{Mor}(\mathscr{A})=\{f_{\mathscr{A}}:A_1\rightarrow A_2\mid A_1\in\mathrm{Obj}(\mathscr{A}),A_2\in\mathrm{Obj}(\mathscr{A})\}\]
から, 対象関数$F_{obj}(\mathscr{A},\mathscr{B})(\cdot)$を用いて定まる, 圏$\mathscr{B}$の射の集合
\[
\mathrm{Mor}(\mathscr{A}\rightarrow\mathscr{B})=\{f_{\mathscr{B}}:F_{\rm obj}(\mathscr{A},\mathscr{B})(A_1)\rightarrow
F_{\rm obj}(\mathscr{A},\mathscr{B})(A_2)\mid A_1\in\mathrm{Obj}(\mathscr{A}),A_2\in\mathrm{Obj}(\mathscr{A})\}
\]
への写像$F_{\rm mor}(\mathscr{A}\rightarrow\mathscr{B}):\mathrm{Mor}(\mathscr{A})\rightarrow\mathrm{Mor}(\mathscr{A}\rightarrow\mathscr{B})$
が次の条件を満たすとき, これを
{\bf 圏$\mathscr{A}$から圏$\mathscr{B}$への
射関数 function on morphisms\index{しゃかんすう@射関数}
}という.
\begin{enumerate}
\item 任意の$A\in\mathrm{Obj}(\mathscr{A})$に対して
\[
F_{\rm mor}(\mathscr{A}\rightarrow\mathscr{B})
(1_A)=1_{F_{\rm mor}(\mathscr{A}\rightarrow\mathscr{B})(A)}\]
が成り立つ.
\item 任意の$(A_1,A_2,A_3)\in \mathrm{Obj}(\mathscr{A})^3$
と$f_{1,2}:A_1\rightarrow A_2, f_{2,3}:A_2\rightarrow A_3$に対して
\[
F_{\mathrm{Mor}}(\mathscr{A}\rightarrow\mathscr{B})(f_{2,3}\circ f_{1,2})
=F_{\mathrm{Mor}}(f_{2,3})\circ F_{\mathrm{Mod}}(f_{1,2}
\]
が成り立つ.
\end{enumerate}
\end{Def}
\begin{Notation}
圏$\mathscr{A}$から圏$\mathscr{B}$への射関数を
$F_{\rm mor}(\mathscr{A}\rightarrow\mathscr{B})$で表す.
\end{Notation}
\begin{Def}
対象関数$F_{\rm obj}(\mathscr{A},\mathscr{B})$と
射関数$F_{\rm mor}(\mathscr{A}\rightarrow\mathscr{B})$の組を{\bf 圏$\mathscr{A}$から圏$\mathscr{B}$への
関手 functor\index{かんしゅ@関手}
}という.
\end{Def}
\begin{Notation}
圏$\mathscr{A}$から圏$\mathscr{B}$への関手$F$を
$F:\mathscr{A}\rightarrow\mathscr{B}$で表す.
\end{Notation}
\begin{Def}
圏$\mathscr{C}$から圏$\mathscr{C}$への関手を{\bf 圏$\mathscr{C}$に関する
恒等関手 identity functor\index{こうとうかんしゅ@恒等関手}
}という.
\end{Def}
\begin{Notation}
圏$\mathscr{C}$に関する恒等関手を$\mathrm{Id}(\mathscr{C})$で表す.
\end{Notation}
\begin{Def}
関手$F:\mathscr{A}\rightarrow\mathscr{B},G:\mathscr{B}\rightarrow\mathscr{C}$に対して
\[F_{\rm obj}(\mathscr{B},\mathscr{C})\circ F_{\rm obj}(\mathscr{A},\mathscr{B}\]
を対象関数とし.
\[F_{\rm mor}(\mathscr{B}\rightarrow\mathscr{C})\circ F_{\rm mor}(\mathscr{A}\rightarrow\mathscr{A})\]
を射関数とする関手
を{\bf 関手$F:\mathscr{A}\rightarrow\mathscr{B},G:\mathscr{B}\rightarrow\mathscr{C}$の合成 composition}という.
\end{Def}
\begin{Notation}
関手$F:\mathscr{A}\rightarrow\mathscr{B},G:\mathscr{B}\rightarrow\mathscr{C}$の合成を$G\circ F$で表す.
\end{Notation}
\begin{comment}
\begin{example}
順序を保存する写像
\end{example}
\end{comment}
\begin{comment}
\begin{example}
圏と見做した順序集合間の簡単な関手の例
\end{example}
\end{comment}
\begin{comment}
*************************
\begin{example}
$n$次ホモロジー関手
\end{example}
***************************
\end{comment}

\subsection{反変関手}
\begin{Def}
関手$F:\mathscr{A}^{\mathrm{op}}\rightarrow\mathscr{B}$
を{\bf 圏$\mathscr{A}$から圏$\mathscr{B}$への
反変関手 contravariant functor\index{はんへんかんしゅ@反変関手}
}という.
\end{Def}
\subsection{定数関手}
\begin{Def}
任意の$A\in\mathrm{Obj}(\mathscr{A})$を唯一の$B_0\in\mathrm{Obj}(\mathscr{B})$に写し,
任意の射$f\in\mathrm{Mor}(\mathscr{A})$を恒等射$1_{B_0}\in\mathrm{Mor}(\mathscr{B})$に写す関手を
{\bf 圏$\mathscr{A}$から圏$\mathscr{B}$への
定数関手 constant functor\index{ていすうかんしゅ@定数関手}
}という
\end{Def}

\subsection{忠実関手と充満関手}
\begin{Def}
関手$F:\mathscr{A}\rightarrow\mathscr{B}$に関して,
写像
\[f:\{(A_1,A_2)\mid A_1\in\mathrm{Obj}(\mathscr{A}),A_2\in\mathrm{Obj}(\mathscr{A})\}\rightarrow\{(F(A_1),F(A_2))\mid A_1\in\mathrm{Obj}(\mathscr{A}),A_2\in\mathrm{Obj}(\mathscr{A})\}\]
が単射となっているとき,
{\bf 関手$F$は
忠実 faithful\index{ちゅうじつ@忠実}
である}という.
\end{Def}
\begin{Def}
関手$F:\mathscr{A}\rightarrow\mathscr{B}$に関して,
写像
\[f:\{(A_1,A_2)\mid A_1\in\mathrm{Obj}(\mathscr{A}),A_2\in\mathrm{Obj}(\mathscr{A})\}\rightarrow\{(F(A_1),F(A_2))\mid A_1\in\mathrm{Obj}(\mathscr{A}),A_2\in\mathrm{Obj}(\mathscr{A})\}\]
が全射となっているとき,
{\bf 関手$F$は
充満 full\index{じゅうまん@充満} 
である}という.
\end{Def}

\begin{Def}
関手$F$が忠実かつ充満であるとき
{\bf 関手$F$は
充満忠実 full and faithful\index{じゅうまんちゅうじつ@充満忠実}
である}という.
\end{Def}

\begin{Def}
圏$\mathscr{A}$が圏$\mathscr{A}$の部分圏であり, 関手$F:\mathscr{A}\rightarrow\mathscr{B}$が充満であるとき,
{\bf 圏$\mathscr{A}$は圏 $\mathscr{B}$の充満部分圏である}という.
\end{Def}

\begin{comment}
\begin{example}
...充満忠実である.
\end{example}
\begin{example}
...忠実だが充満でない
\end{example}
\begin{example}
充満だが忠実でない
\end{example}
\begin{example}
複素数...

...

...忠実だが充満でない. (例1.30)
\end{example}
\end{comment}
\begin{comment}
\subsection{埋め込み関手}
\subsection{忘却関手}
\end{comment}
\begin{comment}
\section{Hask上の関手}
\subsection{List関手(\S 5に挿入)}



\begin{Prop}
Haskellにおける型構築子\verb|[]|は, HaskからHaskへの対象関数である.
\end{Prop}
\begin{proof}
型構築子\verb|[]|は, 任意の型\verb|A|に対して型\verb|[A]|を対応させるので, 命題が成り立つ.
\end{proof}

型\verb|A|と型\verb|B|および関数\verb|f::A->B|が与えられとき\verb|map f::[A]->[B]|が決定される.
\begin{Prop}
\verb|map|はHaskからHaskへの射関数である.
\end{Prop}
\begin{Prop}
型構築子\verb|[]|と\verb|map|がHaskからHaskへの関手をなす.
\end{Prop}
\begin{proof}
命題...と命題...より明らか.
\end{proof}
型構築子\verb|[]|を{\bf List関手}\index{りすとかんしゅ@List関手}という.
\subsubsection{Maybe関手}
Haskellにおける型構築子\verb|Maybe|は...

...

型構築子\verb|Maybe|は{\bf Maybe関手}\index{めいびーかんしゅ@Maybe関手}
と呼ばれる.
\subsubsection{Tree関手}
一般に木構造を生成する型構築子は関手にできる. これを{\bf Tree関手}と呼ぶ.

\lstinputlisting{../hssrc/Tree.hs}
\section{2変数の関手}
{\bf Hom関手}
\lstinputlisting{../hssrc/homfunctors.hs}
\section{型クラスとHaskの部分圏}
{\bf ソート関手}
\lstinputlisting{../hssrc/sort.hs}
\end{comment}



\chapter{関手}
\section{関手}
\begin{Def}
写像$f:\mathrm{Obj}(\mathscr{A})\rightarrow\mathrm{Obj}(\mathscr{B})$を
{\bf 圏$\mathscr{A}$から圏$\mathscr{B}$への
対象関数 function on objects\index{たいしょうかんすう@対象関数}
}という.
\end{Def}
\begin{Notation}
圏$\mathscr{A}$から圏$\mathscr{B}$への対象関数を$F_{\rm obj}(\mathscr{A},\mathscr{B})$で表す.
\end{Notation}
\begin{Def}
圏$\mathscr{A}$の射全体の集合\[\mathrm{Mor}(\mathscr{A})=\{f_{\mathscr{A}}:A_1\rightarrow A_2\mid A_1\in\mathrm{Obj}(\mathscr{A}),A_2\in\mathrm{Obj}(\mathscr{A})\}\]
から, 対象関数$F_{obj}(\mathscr{A},\mathscr{B})(\cdot)$を用いて定まる, 圏$\mathscr{B}$の射の集合
\[
\mathrm{Mor}(\mathscr{A}\rightarrow\mathscr{B})=\{f_{\mathscr{B}}:F_{\rm obj}(\mathscr{A},\mathscr{B})(A_1)\rightarrow
F_{\rm obj}(\mathscr{A},\mathscr{B})(A_2)\mid A_1\in\mathrm{Obj}(\mathscr{A}),A_2\in\mathrm{Obj}(\mathscr{A})\}
\]
への写像$F_{\rm mor}(\mathscr{A}\rightarrow\mathscr{B}):\mathrm{Mor}(\mathscr{A})\rightarrow\mathrm{Mor}(\mathscr{A}\rightarrow\mathscr{B})$
が次の条件を満たすとき, これを
{\bf 圏$\mathscr{A}$から圏$\mathscr{B}$への
射関数 function on morphisms\index{しゃかんすう@射関数}
}という.
\begin{enumerate}
\item 任意の$A\in\mathrm{Obj}(\mathscr{A})$に対して
\[
F_{\rm mor}(\mathscr{A}\rightarrow\mathscr{B})
(1_A)=1_{F_{\rm mor}(\mathscr{A}\rightarrow\mathscr{B})(A)}\]
が成り立つ.
\item 任意の$(A_1,A_2,A_3)\in \mathrm{Obj}(\mathscr{A})^3$
と$f_{1,2}:A_1\rightarrow A_2, f_{2,3}:A_2\rightarrow A_3$に対して
\[
F_{\mathrm{Mor}}(\mathscr{A}\rightarrow\mathscr{B})(f_{2,3}\circ f_{1,2})
=F_{\mathrm{Mor}}(f_{2,3})\circ F_{\mathrm{Mod}}(f_{1,2}
\]
が成り立つ.
\end{enumerate}
\end{Def}
\begin{Notation}
圏$\mathscr{A}$から圏$\mathscr{B}$への射関数を
$F_{\rm mor}(\mathscr{A}\rightarrow\mathscr{B})$で表す.
\end{Notation}
\begin{Def}
対象関数$F_{\rm obj}(\mathscr{A},\mathscr{B})$と
射関数$F_{\rm mor}(\mathscr{A}\rightarrow\mathscr{B})$の組を{\bf 圏$\mathscr{A}$から圏$\mathscr{B}$への
関手 functor\index{かんしゅ@関手}
}という.
\end{Def}
\begin{Notation}
圏$\mathscr{A}$から圏$\mathscr{B}$への関手$F$を
$F:\mathscr{A}\rightarrow\mathscr{B}$で表す.
\end{Notation}
\begin{Def}
圏$\mathscr{C}$から圏$\mathscr{C}$への関手を{\bf 圏$\mathscr{C}$に関する
恒等関手 identity functor\index{こうとうかんしゅ@恒等関手}
}という.
\end{Def}
\begin{Notation}
圏$\mathscr{C}$に関する恒等関手を$\mathrm{Id}(\mathscr{C})$で表す.
\end{Notation}
\begin{Def}
関手$F:\mathscr{A}\rightarrow\mathscr{B},G:\mathscr{B}\rightarrow\mathscr{C}$に対して
\[F_{\rm obj}(\mathscr{B},\mathscr{C})\circ F_{\rm obj}(\mathscr{A},\mathscr{B}\]
を対象関数とし.
\[F_{\rm mor}(\mathscr{B}\rightarrow\mathscr{C})\circ F_{\rm mor}(\mathscr{A}\rightarrow\mathscr{A})\]
を射関数とする関手
を{\bf 関手$F:\mathscr{A}\rightarrow\mathscr{B},G:\mathscr{B}\rightarrow\mathscr{C}$の合成 composition}という.
\end{Def}
\begin{Notation}
関手$F:\mathscr{A}\rightarrow\mathscr{B},G:\mathscr{B}\rightarrow\mathscr{C}$の合成を$G\circ F$で表す.
\end{Notation}
\begin{comment}
\begin{example}
順序を保存する写像
\end{example}
\end{comment}
\begin{comment}
\begin{example}
圏と見做した順序集合間の簡単な関手の例
\end{example}
\end{comment}
\begin{comment}
*************************
\begin{example}
$n$次ホモロジー関手
\end{example}
***************************
\end{comment}

\subsection{反変関手}
\begin{Def}
関手$F:\mathscr{A}^{\mathrm{op}}\rightarrow\mathscr{B}$
を{\bf 圏$\mathscr{A}$から圏$\mathscr{B}$への
反変関手 contravariant functor\index{はんへんかんしゅ@反変関手}
}という.
\end{Def}
\subsection{定数関手}
\begin{Def}
任意の$A\in\mathrm{Obj}(\mathscr{A})$を唯一の$B_0\in\mathrm{Obj}(\mathscr{B})$に写し,
任意の射$f\in\mathrm{Mor}(\mathscr{A})$を恒等射$1_{B_0}\in\mathrm{Mor}(\mathscr{B})$に写す関手を
{\bf 圏$\mathscr{A}$から圏$\mathscr{B}$への
定数関手 constant functor\index{ていすうかんしゅ@定数関手}
}という
\end{Def}

\subsection{忠実関手と充満関手}
\begin{Def}
関手$F:\mathscr{A}\rightarrow\mathscr{B}$に関して,
写像
\[f:\{(A_1,A_2)\mid A_1\in\mathrm{Obj}(\mathscr{A}),A_2\in\mathrm{Obj}(\mathscr{A})\}\rightarrow\{(F(A_1),F(A_2))\mid A_1\in\mathrm{Obj}(\mathscr{A}),A_2\in\mathrm{Obj}(\mathscr{A})\}\]
が単射となっているとき,
{\bf 関手$F$は
忠実 faithful\index{ちゅうじつ@忠実}
である}という.
\end{Def}
\begin{Def}
関手$F:\mathscr{A}\rightarrow\mathscr{B}$に関して,
写像
\[f:\{(A_1,A_2)\mid A_1\in\mathrm{Obj}(\mathscr{A}),A_2\in\mathrm{Obj}(\mathscr{A})\}\rightarrow\{(F(A_1),F(A_2))\mid A_1\in\mathrm{Obj}(\mathscr{A}),A_2\in\mathrm{Obj}(\mathscr{A})\}\]
が全射となっているとき,
{\bf 関手$F$は
充満 full\index{じゅうまん@充満} 
である}という.
\end{Def}

\begin{Def}
関手$F$が忠実かつ充満であるとき
{\bf 関手$F$は
充満忠実 full and faithful\index{じゅうまんちゅうじつ@充満忠実}
である}という.
\end{Def}

\begin{Def}
圏$\mathscr{A}$が圏$\mathscr{A}$の部分圏であり, 関手$F:\mathscr{A}\rightarrow\mathscr{B}$が充満であるとき,
{\bf 圏$\mathscr{A}$は圏 $\mathscr{B}$の充満部分圏である}という.
\end{Def}

\begin{comment}
\begin{example}
...充満忠実である.
\end{example}
\begin{example}
...忠実だが充満でない
\end{example}
\begin{example}
充満だが忠実でない
\end{example}
\begin{example}
複素数...

...

...忠実だが充満でない. (例1.30)
\end{example}
\end{comment}
\begin{comment}
\subsection{埋め込み関手}
\subsection{忘却関手}
\end{comment}
\begin{comment}
\section{Hask上の関手}
\subsection{List関手(\S 5に挿入)}



\begin{Prop}
Haskellにおける型構築子\verb|[]|は, HaskからHaskへの対象関数である.
\end{Prop}
\begin{proof}
型構築子\verb|[]|は, 任意の型\verb|A|に対して型\verb|[A]|を対応させるので, 命題が成り立つ.
\end{proof}

型\verb|A|と型\verb|B|および関数\verb|f::A->B|が与えられとき\verb|map f::[A]->[B]|が決定される.
\begin{Prop}
\verb|map|はHaskからHaskへの射関数である.
\end{Prop}
\begin{Prop}
型構築子\verb|[]|と\verb|map|がHaskからHaskへの関手をなす.
\end{Prop}
\begin{proof}
命題...と命題...より明らか.
\end{proof}
型構築子\verb|[]|を{\bf List関手}\index{りすとかんしゅ@List関手}という.
\subsubsection{Maybe関手}
Haskellにおける型構築子\verb|Maybe|は...

...

型構築子\verb|Maybe|は{\bf Maybe関手}\index{めいびーかんしゅ@Maybe関手}
と呼ばれる.
\subsubsection{Tree関手}
一般に木構造を生成する型構築子は関手にできる. これを{\bf Tree関手}と呼ぶ.

\lstinputlisting{../hssrc/Tree.hs}
\section{2変数の関手}
{\bf Hom関手}
\lstinputlisting{../hssrc/homfunctors.hs}
\section{型クラスとHaskの部分圏}
{\bf ソート関手}
\lstinputlisting{../hssrc/sort.hs}
\end{comment}



\chapter{自然変換}
\section{関手}
\begin{Def}
写像$f:\mathrm{Obj}(\mathscr{A})\rightarrow\mathrm{Obj}(\mathscr{B})$を
{\bf 圏$\mathscr{A}$から圏$\mathscr{B}$への
対象関数 function on objects\index{たいしょうかんすう@対象関数}
}という.
\end{Def}
\begin{Notation}
圏$\mathscr{A}$から圏$\mathscr{B}$への対象関数を$F_{\rm obj}(\mathscr{A},\mathscr{B})$で表す.
\end{Notation}
\begin{Def}
圏$\mathscr{A}$の射全体の集合\[\mathrm{Mor}(\mathscr{A})=\{f_{\mathscr{A}}:A_1\rightarrow A_2\mid A_1\in\mathrm{Obj}(\mathscr{A}),A_2\in\mathrm{Obj}(\mathscr{A})\}\]
から, 対象関数$F_{obj}(\mathscr{A},\mathscr{B})(\cdot)$を用いて定まる, 圏$\mathscr{B}$の射の集合
\[
\mathrm{Mor}(\mathscr{A}\rightarrow\mathscr{B})=\{f_{\mathscr{B}}:F_{\rm obj}(\mathscr{A},\mathscr{B})(A_1)\rightarrow
F_{\rm obj}(\mathscr{A},\mathscr{B})(A_2)\mid A_1\in\mathrm{Obj}(\mathscr{A}),A_2\in\mathrm{Obj}(\mathscr{A})\}
\]
への写像$F_{\rm mor}(\mathscr{A}\rightarrow\mathscr{B}):\mathrm{Mor}(\mathscr{A})\rightarrow\mathrm{Mor}(\mathscr{A}\rightarrow\mathscr{B})$
が次の条件を満たすとき, これを
{\bf 圏$\mathscr{A}$から圏$\mathscr{B}$への
射関数 function on morphisms\index{しゃかんすう@射関数}
}という.
\begin{enumerate}
\item 任意の$A\in\mathrm{Obj}(\mathscr{A})$に対して
\[
F_{\rm mor}(\mathscr{A}\rightarrow\mathscr{B})
(1_A)=1_{F_{\rm mor}(\mathscr{A}\rightarrow\mathscr{B})(A)}\]
が成り立つ.
\item 任意の$(A_1,A_2,A_3)\in \mathrm{Obj}(\mathscr{A})^3$
と$f_{1,2}:A_1\rightarrow A_2, f_{2,3}:A_2\rightarrow A_3$に対して
\[
F_{\mathrm{Mor}}(\mathscr{A}\rightarrow\mathscr{B})(f_{2,3}\circ f_{1,2})
=F_{\mathrm{Mor}}(f_{2,3})\circ F_{\mathrm{Mod}}(f_{1,2}
\]
が成り立つ.
\end{enumerate}
\end{Def}
\begin{Notation}
圏$\mathscr{A}$から圏$\mathscr{B}$への射関数を
$F_{\rm mor}(\mathscr{A}\rightarrow\mathscr{B})$で表す.
\end{Notation}
\begin{Def}
対象関数$F_{\rm obj}(\mathscr{A},\mathscr{B})$と
射関数$F_{\rm mor}(\mathscr{A}\rightarrow\mathscr{B})$の組を{\bf 圏$\mathscr{A}$から圏$\mathscr{B}$への
関手 functor\index{かんしゅ@関手}
}という.
\end{Def}
\begin{Notation}
圏$\mathscr{A}$から圏$\mathscr{B}$への関手$F$を
$F:\mathscr{A}\rightarrow\mathscr{B}$で表す.
\end{Notation}
\begin{Def}
圏$\mathscr{C}$から圏$\mathscr{C}$への関手を{\bf 圏$\mathscr{C}$に関する
恒等関手 identity functor\index{こうとうかんしゅ@恒等関手}
}という.
\end{Def}
\begin{Notation}
圏$\mathscr{C}$に関する恒等関手を$\mathrm{Id}(\mathscr{C})$で表す.
\end{Notation}
\begin{Def}
関手$F:\mathscr{A}\rightarrow\mathscr{B},G:\mathscr{B}\rightarrow\mathscr{C}$に対して
\[F_{\rm obj}(\mathscr{B},\mathscr{C})\circ F_{\rm obj}(\mathscr{A},\mathscr{B}\]
を対象関数とし.
\[F_{\rm mor}(\mathscr{B}\rightarrow\mathscr{C})\circ F_{\rm mor}(\mathscr{A}\rightarrow\mathscr{A})\]
を射関数とする関手
を{\bf 関手$F:\mathscr{A}\rightarrow\mathscr{B},G:\mathscr{B}\rightarrow\mathscr{C}$の合成 composition}という.
\end{Def}
\begin{Notation}
関手$F:\mathscr{A}\rightarrow\mathscr{B},G:\mathscr{B}\rightarrow\mathscr{C}$の合成を$G\circ F$で表す.
\end{Notation}
\begin{comment}
\begin{example}
順序を保存する写像
\end{example}
\end{comment}
\begin{comment}
\begin{example}
圏と見做した順序集合間の簡単な関手の例
\end{example}
\end{comment}
\begin{comment}
*************************
\begin{example}
$n$次ホモロジー関手
\end{example}
***************************
\end{comment}

\subsection{反変関手}
\begin{Def}
関手$F:\mathscr{A}^{\mathrm{op}}\rightarrow\mathscr{B}$
を{\bf 圏$\mathscr{A}$から圏$\mathscr{B}$への
反変関手 contravariant functor\index{はんへんかんしゅ@反変関手}
}という.
\end{Def}
\subsection{定数関手}
\begin{Def}
任意の$A\in\mathrm{Obj}(\mathscr{A})$を唯一の$B_0\in\mathrm{Obj}(\mathscr{B})$に写し,
任意の射$f\in\mathrm{Mor}(\mathscr{A})$を恒等射$1_{B_0}\in\mathrm{Mor}(\mathscr{B})$に写す関手を
{\bf 圏$\mathscr{A}$から圏$\mathscr{B}$への
定数関手 constant functor\index{ていすうかんしゅ@定数関手}
}という
\end{Def}

\subsection{忠実関手と充満関手}
\begin{Def}
関手$F:\mathscr{A}\rightarrow\mathscr{B}$に関して,
写像
\[f:\{(A_1,A_2)\mid A_1\in\mathrm{Obj}(\mathscr{A}),A_2\in\mathrm{Obj}(\mathscr{A})\}\rightarrow\{(F(A_1),F(A_2))\mid A_1\in\mathrm{Obj}(\mathscr{A}),A_2\in\mathrm{Obj}(\mathscr{A})\}\]
が単射となっているとき,
{\bf 関手$F$は
忠実 faithful\index{ちゅうじつ@忠実}
である}という.
\end{Def}
\begin{Def}
関手$F:\mathscr{A}\rightarrow\mathscr{B}$に関して,
写像
\[f:\{(A_1,A_2)\mid A_1\in\mathrm{Obj}(\mathscr{A}),A_2\in\mathrm{Obj}(\mathscr{A})\}\rightarrow\{(F(A_1),F(A_2))\mid A_1\in\mathrm{Obj}(\mathscr{A}),A_2\in\mathrm{Obj}(\mathscr{A})\}\]
が全射となっているとき,
{\bf 関手$F$は
充満 full\index{じゅうまん@充満} 
である}という.
\end{Def}

\begin{Def}
関手$F$が忠実かつ充満であるとき
{\bf 関手$F$は
充満忠実 full and faithful\index{じゅうまんちゅうじつ@充満忠実}
である}という.
\end{Def}

\begin{Def}
圏$\mathscr{A}$が圏$\mathscr{A}$の部分圏であり, 関手$F:\mathscr{A}\rightarrow\mathscr{B}$が充満であるとき,
{\bf 圏$\mathscr{A}$は圏 $\mathscr{B}$の充満部分圏である}という.
\end{Def}

\begin{comment}
\begin{example}
...充満忠実である.
\end{example}
\begin{example}
...忠実だが充満でない
\end{example}
\begin{example}
充満だが忠実でない
\end{example}
\begin{example}
複素数...

...

...忠実だが充満でない. (例1.30)
\end{example}
\end{comment}
\begin{comment}
\subsection{埋め込み関手}
\subsection{忘却関手}
\end{comment}
\begin{comment}
\section{Hask上の関手}
\subsection{List関手(\S 5に挿入)}



\begin{Prop}
Haskellにおける型構築子\verb|[]|は, HaskからHaskへの対象関数である.
\end{Prop}
\begin{proof}
型構築子\verb|[]|は, 任意の型\verb|A|に対して型\verb|[A]|を対応させるので, 命題が成り立つ.
\end{proof}

型\verb|A|と型\verb|B|および関数\verb|f::A->B|が与えられとき\verb|map f::[A]->[B]|が決定される.
\begin{Prop}
\verb|map|はHaskからHaskへの射関数である.
\end{Prop}
\begin{Prop}
型構築子\verb|[]|と\verb|map|がHaskからHaskへの関手をなす.
\end{Prop}
\begin{proof}
命題...と命題...より明らか.
\end{proof}
型構築子\verb|[]|を{\bf List関手}\index{りすとかんしゅ@List関手}という.
\subsubsection{Maybe関手}
Haskellにおける型構築子\verb|Maybe|は...

...

型構築子\verb|Maybe|は{\bf Maybe関手}\index{めいびーかんしゅ@Maybe関手}
と呼ばれる.
\subsubsection{Tree関手}
一般に木構造を生成する型構築子は関手にできる. これを{\bf Tree関手}と呼ぶ.

\lstinputlisting{../hssrc/Tree.hs}
\section{2変数の関手}
{\bf Hom関手}
\lstinputlisting{../hssrc/homfunctors.hs}
\section{型クラスとHaskの部分圏}
{\bf ソート関手}
\lstinputlisting{../hssrc/sort.hs}
\end{comment}



\chapter{関手圏}
\section{関手圏}
\begin{Prop}
...は圏である.
\end{Prop}
\begin{Def}
{\bf $\mathscr{A}$から$\mathscr{B}$への関手圏}という.
\end{Def}
\begin{Notation}
$\mathscr{A}$から$\mathscr{B}$への関手圏を$[\mathscr{A},\mathscr{B}]$
で表す\end{Notation}
\begin{Def}
関手圏における同型射を自然同型という.
\bf{自然同型}
\end{Def}
\begin{Prop}
自然変換...が自然同型であること,

...が同型者であることは同値である.
\end{Prop}

\chapter{圏同値}
\section{圏同値}
\begin{Def}
\bf{圏同値}
\end{Def}
\begin{comment}
\section{Haskにおける自然同型}
\subsection{mirror関数}
Tree関手からTree関手への自然同型
\subsection{Maybe関手とEither()関手の間の自然同型}
\section{まとめ}
aa
\end{comment}
\printindex
\end{document}
