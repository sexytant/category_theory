\section{関手}
\begin{Def}
写像$f:\mathrm{Obj}(\mathscr{A})\rightarrow\mathrm{Obj}(\mathscr{B})$を
{\bf 圏$\mathscr{A}$から圏$\mathscr{B}$への対象関数 function on objects}という.
\end{Def}
\begin{Notation}
圏$\mathscr{A}$から圏$\mathscr{B}$への対象関数を$F_{\rm obj}(\mathscr{A},\mathscr{B})$で表す.
\end{Notation}
\begin{Def}
圏$\mathscr{A}$の射全体の集合\[\mathrm{Mor}(\mathscr{A})=\{f_{\mathscr{A}}:A_1\rightarrow A_2\mid A_1\in\mathrm{Obj}(\mathscr{A}),A_2\in\mathrm{Obj}(\mathscr{A})\}\]
から, 対象関数$F_{obj}(\mathscr{A},\mathscr{B})(\cdot)$を用いて定まる, 圏$\mathscr{B}$の射の集合
\[
\mathrm{Mor}(\mathscr{A}\rightarrow\mathscr{B})=\{f_{\mathscr{B}}:F_{\rm obj}(\mathscr{A},\mathscr{B})(A_1)\rightarrow
F_{\rm obj}(\mathscr{A},\mathscr{B})(A_2)\mid A_1\in\mathrm{Obj}(\mathscr{A}),A_2\in\mathrm{Obj}(\mathscr{A})\}
\]
への写像$F_{\rm mor}(\mathscr{A}\rightarrow\mathscr{B}):\mathrm{Mor}(\mathscr{A})\rightarrow\mathrm{Mor}(\mathscr{A}\rightarrow\mathscr{B})$
が次の条件を満たすとき, これを
{\bf 圏$\mathscr{A}$から圏$\mathscr{B}$への射関数 function on morphisms}という.
\begin{enumerate}
\item 任意の$A\in\mathrm{Obj}(\mathscr{A})$に対して
\[
F_{\rm mor}(\mathscr{A}\rightarrow\mathscr{B})
(1_A)=1_{F_{\rm mor}(\mathscr{A}\rightarrow\mathscr{B})(A)}\]
が成り立つ.
\item 任意の$(A_1,A_2,A_3)\in \mathrm{Obj}(\mathscr{A})^3$
と$f_{1,2}:A_1\rightarrow A_2, f_{2,3}:A_2\rightarrow A_3$に対して
\[
F_{\mathrm{Mor}}(\mathscr{A}\rightarrow\mathscr{B})(f_{2,3}\circ f_{1,2})
=F_{\mathrm{Mor}}(f_{2,3})\circ F_{\mathrm{Mod}}(f_{1,2}
\]
が成り立つ.
\end{enumerate}
\end{Def}
\begin{Notation}
圏$\mathscr{A}$から圏$\mathscr{B}$への射関数を
$F_{\rm mor}(\mathscr{A}\rightarrow\mathscr{B})$で表す.
\end{Notation}
\begin{Def}
対象関数$F_{\rm obj}(\mathscr{A},\mathscr{B})$と
射関数$F_{\rm mor}(\mathscr{A}\rightarrow\mathscr{B})$の組を{\bf 圏$\mathscr{A}$から圏$\mathscr{B}$への関手 functor}という.
\end{Def}
\begin{Notation}
圏$\mathscr{A}$から圏$\mathscr{B}$への関手$F$を
$F:\mathscr{A}\rightarrow\mathscr{B}$で表す.
\end{Notation}
\begin{Def}
圏$\mathscr{C}$から圏$\mathscr{C}$への関手を{\bf 圏$\mathscr{C}$に関する恒等関手 identity functor}という.
\end{Def}
\begin{Notation}
圏$\mathscr{C}$に関する恒等関手を$\mathrm{Id}(\mathscr{C})$で表す.
\end{Notation}
\begin{Def}
関手$F:\mathscr{A}\rightarrow\mathscr{B},G:\mathscr{B}\rightarrow\mathscr{C}$に対して
\[F_{\rm obj}(\mathscr{B},\mathscr{C})\circ F_{\rm obj}(\mathscr{A},\mathscr{B}\]
を対象関数とし.
\[F_{\rm mor}(\mathscr{B}\rightarrow\mathscr{C})\circ F_{\rm mor}(\mathscr{A}\rightarrow\mathscr{A})\]
を射関数とする関手
を{\bf 関手$F:\mathscr{A}\rightarrow\mathscr{B},G:\mathscr{B}\rightarrow\mathscr{C}$の合成 composition}という.
\end{Def}
\begin{Notation}
関手$F:\mathscr{A}\rightarrow\mathscr{B},G:\mathscr{B}\rightarrow\mathscr{C}$の合成を$G\circ F$で表す.
\end{Notation}
\begin{comment}
\begin{example}
順序を保存する写像
\end{example}
\end{comment}
\begin{comment}
\begin{example}
圏と見做した順序集合間の簡単な関手の例
\end{example}
\end{comment}
\begin{comment}
*************************
\begin{example}
$n$次ホモロジー関手
\end{example}
***************************
\end{comment}

\subsection{反変関手}
\begin{Def}
関手$F:\mathscr{A}^{\mathrm{op}}\rightarrow\mathscr{B}$
を{\bf 圏$\mathscr{A}$から圏$\mathscr{B}$への反変関手 contravariant functor}という.
\end{Def}
\subsection{定数関手}
\begin{Def}
任意の$A\in\mathrm{Obj}(\mathscr{A})$を唯一の$B_0\in\mathrm{Obj}(\mathscr{B})$に写し,
任意の射$f\in\mathrm{Mor}(\mathscr{A})$を恒等射$1_{B_0}\in\mathrm{Mor}(\mathscr{B})$に写す関手を
{\bf 圏$\mathscr{A}$から圏$\mathscr{B}$への定数関手 constant functor}\index{ていすうかんしゅ@定数関手}という
\end{Def}

\subsection{忠実関手と充満関手}
\begin{Def}
関手$F:\mathscr{A}\rightarrow\mathscr{B}$に関して,
写像
\[f:\{(A_1,A_2)\mid A_1\in\mathrm{Obj}(\mathscr{A}),A_2\in\mathrm{Obj}(\mathscr{A})\}\rightarrow\{(F(A_1),F(A_2))\mid A_1\in\mathrm{Obj}(\mathscr{A}),A_2\in\mathrm{Obj}(\mathscr{A})\}\]
が単射となっているとき,
{\bf 関手$F$は忠実 faithfulである}という.
\end{Def}
\begin{Def}
関手$F:\mathscr{A}\rightarrow\mathscr{B}$に関して,
写像
\[f:\{(A_1,A_2)\mid A_1\in\mathrm{Obj}(\mathscr{A}),A_2\in\mathrm{Obj}(\mathscr{A})\}\rightarrow\{(F(A_1),F(A_2))\mid A_1\in\mathrm{Obj}(\mathscr{A}),A_2\in\mathrm{Obj}(\mathscr{A})\}\]
が全射となっているとき,
{\bf 関手$F$は充満 full である}という.
\end{Def}

\begin{Def}
関手$F$が忠実かつ充満であるとき
{\bf 関手$F$は充満忠実 full and faithfulである}という.
\end{Def}
\begin{comment}
\begin{Def}
圏$\mathscr{A}$が圏$\mathscr{A}$の部分圏であり, 関手$F:\mathscr{A}\rightarrow\mathscr{B}$が充満であるとき,
{\bf 圏$\mathscr{A}$は圏 $\mathscr{B}$の充満部分圏である}という.
\end{Def}
\end{comment}
\begin{comment}
\begin{example}
...充満忠実である.
\end{example}
\begin{example}
...忠実だが充満でない
\end{example}
\begin{example}
充満だが忠実でない
\end{example}
\begin{example}
複素数...

...

...忠実だが充満でない. (例1.30)
\end{example}
\end{comment}
\begin{comment}
\subsection{埋め込み関手}
\subsection{忘却関手}
\end{comment}



