\section{関手}
\begin{Def}
圏$\mathscr{A}$の対象$\mathrm{Obj}(\mathscr{A})$から圏$\mathscr{B}$の対象$\mathrm{Obj}(\mathscr{B})$への写像を
{\bf 圏$\mathscr{A}$から圏$\mathscr{B}$への対象関数 function on objects}という.
\end{Def}
\begin{Def}
圏$\mathscr{A}$に関する射の集合\[\mathrm{Map}_{\mathscr{A}}=\{f_{\mathscr{A}}:A_1\rightarrow A_2\mid A_1\in\mathrm{Obj}(\mathscr{A}),A_2\in\mathrm{Obj}(\mathscr{A})\}\]
から, その各元に対して, 
圏$\mathscr{A}$から圏$\mathscr{B}$への対象関数$F_{\rm obj}(\mathscr{A},\mathscr{B})(\cdot)$を作用させて定まる射の集合
\[
\mathrm{Map_\mathscr{B}}=\{f_{\mathscr{B}}:F_{\rm obj}(\mathscr{A},\mathscr{B})(A_1),
F_{\rm obj}(\mathscr{A},\mathscr{B})(A_2)\mid A_1\in\mathrm{Obj}(\mathscr{A}),A_2\in\mathrm{Obj}(\mathscr{A})\}
\]
への写像$F_{\rm Mor}(\mathscr{A},\mathscr{B}):\mathrm{Map}_{\mathscr{A}}\rightarrow\mathrm{Map}_{\mathscr{B}}$
を
{\bf 圏$\mathscr{A}$から圏$\mathscr{B}$への射関数 function on morphisms}という.
\end{Def}
\begin{Def}
圏$\mathscr{A},\mathscr{B}$に対する対象関数と射関数の組を{\bf 圏$\mathscr{A}$から圏$\mathscr{B}$への関手 functor}という.
\end{Def}
\begin{Notation}
圏$\mathscr{A}$から圏$\mathscr{B}$への関手$F$を
$F:\mathscr{A}\rightarrow\mathscr{B}$で表す.
\end{Notation}
\begin{comment}
\begin{example}
順序を保存する写像
\end{example}
\end{comment}
\begin{comment}
\begin{example}
圏と見做した順序集合間の簡単な関手の例
\end{example}
\end{comment}
\begin{comment}
*************************
\begin{example}
$n$次ホモロジー関手
\end{example}
***************************
\end{comment}

\subsection{反変関手}
\begin{Def}
圏$\mathscr{A}^{\mathrm{op}}$から圏$\mathscr{B}$への関手
を{\bf 圏$\mathscr{A}$から圏$\mathscr{B}$への反変関手 contravariant functor}という.
\end{Def}
\subsection{定数関手}
\begin{Def}
圏$\mathscr{A}$の任意の対象$A$を圏$\mathscr{B}$のただ一つの対象$B_0$に写し,
$\mathscr{A}$の任意の射$f$を$\mathscr{B}$の恒等射に写す関手を
{\bf 圏$\mathscr{A}$から圏$\mathscr{B}$への定数関手 constant functor}\index{ていすうかんしゅ@定数関手}という
\end{Def}

\subsection{忠実関手と充満関手}
\begin{Def}
圏$\mathscr{A}$
から圏$\mathscr{B}$への関手$F$に関して,
集合$\{(A_1,A_2)\mid A_1,A_2\in\mathrm{Obj}(\mathscr{A})\}$
から
集合$\{(F(A_1),F(A_2))\mid A_1,A_2\mathrm{Obj}(\mathscr{A}))\}$
への写像が単射となっているとき,
{\bf 関手$F$は忠実 faithful}であるという.
\end{Def}
\begin{Def}
圏$\mathscr{A}$から圏$\mathscr{B}$への関手$F$に関して,
集合$\{(A_1,A_2)\mid A_1,A_2\in\mathrm{Obj}(\mathscr{A})\}$
から
集合$\{(F(A_1),F(A_2))\mid A_1,A_2\in\mathrm{Obj}(\mathscr{A}))\}$
への写像が全射となっているとき,
{\bf 関手$F$は充満 full}であるという.
\end{Def}
\begin{Def}圏$\mathscr{A}$から圏$\mathscr{B}$への関手$F$が忠実かつ充満であるとき
{\bf 関手$F$は充満忠実 full and faithfulである}という
\end{Def}
\begin{comment}
\begin{Def}
圏$\mathscr{A}$が圏$\mathscr{A}$の部分圏であり, 関手$F:\mathscr{A}\rightarrow\mathscr{B}$が充満であるとき,
{\bf 圏$\mathscr{A}$は圏 $\mathscr{B}$の充満部分圏である}という.
\end{Def}
\end{comment}
\begin{comment}
\begin{example}
...充満忠実である.
\end{example}
\begin{example}
...忠実だが充満でない
\end{example}
\begin{example}
充満だが忠実でない
\end{example}
\begin{example}
複素数...

...

...忠実だが充満でない. (例1.30)
\end{example}
\end{comment}
\begin{comment}
\subsection{埋め込み関手}
\subsection{忘却関手}
\end{comment}
\begin{comment}
\section{Haskellの型構築子と関手}
\subsubsection{List関手}
Haskellにおける型構築子\verb|[]|は任意の型\verb|A|に対して型\verb|[A]|を対応させる.
これは, HaskからHaskへの対称関数とみなせる.
型\verb|A|と型\verb|B|および関数\verb|f::A->B|が与えられとき\verb|map f::[A]->[B]|が決定される.

...

型構築子\verb|[]|は{\bf List関手}\index{りすとかんしゅ@List関手}と呼ばれる
\subsubsection{Maybe関手}
Haskellにおける型構築子\verb|Maybe|は...

...

型構築子\verb|Maybe|は{\bf Maybe関手}\index{めいびーかんしゅ@Maybe関手}
と呼ばれる.
\subsubsection{Tree関手}
一般に木構造を生成する型構築子は関手にできる. これを{\bf Tree関手}と呼ぶ.

\lstinputlisting{../hssrc/Tree.hs}
\section{2変数の関手}
{\bf Hom関手}
\lstinputlisting{../hssrc/homfunctors.hs}
\section{型クラスとHaskの部分圏}
{\bf ソート関手}
\lstinputlisting{../hssrc/sort.hs}
\end{comment}
