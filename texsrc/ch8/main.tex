\section{圏同値}
\begin{Def}
圏$\mathscr{A},\mathscr{A}'$に対して, 関手$F:\mathscr{A}\rightarrow\mathscr{A}', G:\mathscr{A}\rightarrow\mathscr{A}'$と自然同型$\eta:\mathrm{Id}(\mathscr{A})\rightarrow G\circ F,\epsilon:F\circ G\rightarrow\rm{Id}(\mathscr{A}')$が存在するとき
\bf{圏$\mathscr{A},\mathscr{A}'$は
圏同値 equivalence of categories\index{けんどうち@圏同値}
である}という.
\end{Def}
\begin{Prop}
圏同値は同値関係である.
\end{Prop}

\begin{comment}
\section{Haskにおける自然同型}
\subsection{mirror関数}
Tree関手からTree関手への自然同型
\subsection{Maybe関手とEither()関手の間の自然同型}
\section{まとめ}
aa
\end{comment}

\section{この先レビュー対象外}
命題8.2の証明