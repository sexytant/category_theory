\section{自然変換}
\begin{Def}
圏$\mathscr{A}$から圏$\mathscr{B}$への関手$F,G$について,
\begin{enumerate}
\item $\mathscr{A}$の任意の対象$A$ 
に対して
$\mathscr{B}$の射$\alpha_A:FA\rightarrow GA$
が存在し
\item $\mathscr{A}$の任意の射$f:A_1\rightarrow A_2$
が
\[
\alpha_{A_2}\circ Ff
=Gf\circ\alpha_{A_1}
\]
を満たすとき
\end{enumerate}に対して
{\bf 集合$\alpha=\{\alpha_A\mid A\in\mathrm{Obj}(\mathscr{A})\}$は関手$F$から関手$G$への自然変換 \index{しぜんへんかん@自然変換}natural transformationである}という.
\end{Def}
\begin{Notation}
関手$F$から関手$G$への自然変換$\alpha$を$\alpha:F\rightarrow G$で表す.
\end{Notation}
\begin{comment}
\begin{Notation}
関手$F$から関手$G$への自然変換$\alpha$を次の図表で表す.
\end{Notation}
\end{comment}
\begin{comment}
\section{Haskにおける自然変換}
\subsection{concat}
\subsection{List関手からMaybe関手への自然変換safehead}
\subsection{concatとsafeheadの垂直合成}
\subsection{二分木からリストへのflatten関数}

\section{Haskにおける定数関手}
\subsection{length関数}
\end{comment}