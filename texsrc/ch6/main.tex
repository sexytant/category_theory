\section{自然変換}
\begin{Def}
関手$F:\mathscr{A}\rightarrow\mathscr{B}$と関手$G:\mathscr{A}\rightarrow\mathscr{B}$について,
\begin{enumerate}
\item 任意の$A\in\mathrm{Obj}(\mathscr{A})$に対して
$\mathscr{B}$の射$M_{\mathscr{B}}(A):F(A)\rightarrow G(A)$
が存在する.
\item 任意の$\mathscr{A}$の射$f:A_1\rightarrow A_2$
が
\[
M_{\mathscr{B}}(A_2)\circ F(f)
=G(f)\circ M_{\mathscr{B}}(A_1)
\]
を満たす.
\end{enumerate}
このとき
$\{M_{\mathscr{B}}(A)\mid A\in\mathrm{Obj}(\mathscr{A})\}$を{\bf 関手$F:\mathscr{A}\rightarrow\mathscr{B}$から関手$G:\mathscr{A}\rightarrow\mathscr{B}$への自然変換 \index{しぜんへんかん@自然変換}natural transformation}という.
\end{Def}
\begin{Notation}
関手$F$から関手$G$への自然変換$\mu$を$\mu:F\rightarrow G$で表す.
\end{Notation}
\begin{comment}
\begin{Notation}
関手$F$から関手$G$への自然変換$\alpha$を次の図表で表す.
\end{Notation}
\end{comment}
\begin{comment}
\section{Haskにおける自然変換}
\subsection{concat}
\subsection{List関手からMaybe関手への自然変換safehead}
\subsection{concatとsafeheadの垂直合成}
\subsection{二分木からリストへのflatten関数}

\section{Haskにおける定数関手}
\subsection{length関数}
\end{comment}


