\section{マグマ・半群・群}
\subsection{マグマ}
\begin{Def}
集合$S$に関する写像$\mu:S\times S\rightarrow S$
を{\bf $S$上の二項演算 binary operation $\mu$}という.
\end{Def}
\begin{Def}
集合$M$と集合$M$上の二項演算$\mu$の組$(M,\mu)$を{\bf マグマ magma}という.
\end{Def}
\subsection{半群}
\begin{Def}
マグマ$(G,\mu)$について,
$\mu$が結合律を満たす, すなわち任意の元$g,h,k\in G$に対して\[\mu(g,\mu(h,k))=\mu(\mu(g,h),k)\]
が成り立つとき, {\bf $(G,\mu)$は半群 semigroupである}という.
\end{Def}
\subsection{群}
\begin{Def}
マグマ$(G,\mu)$について, とある元$e\in G$が存在し, 任意の元$g\in G$に対して\[\mu(e,g)=\mu(g,e)=g\]が成り立つとき, $e$を{\bf $G$の単位元 identity element}という.
\end{Def}
\begin{Def}
マグマ$(G,\mu)$について, 任意の元$g\in G$に対して, とある元$h\in G$が存在して \[\mu(g,h)=\mu(h,g)=e\]が成り立つとき, $h$を{\bf $g$の逆元 inverse element }という.
ここで$e$は$G$の単位元である.
\end{Def}
\begin{Notation}
マグマ$(G,\mu)$における元$g\in G$の逆元を$g^{-1}$で表す.
\end{Notation}
\begin{Def}
半群$(G,\mu)$が単位元をもち, かつ任意の元に対して逆元が存在するとき\bf{$(G,\mu)$は群 groupである}という
\end{Def}
\begin{Def}
群$(G_1,\mu_1)$から群$(G_2,\mu_2)$への写像$f$が任意の$G_1$の元$g, g'$について \[f(\mu_1(g,g')) = \mu_2(f(g),f(g'))\] を満たすとき、{\bf $f$は$(G_1,\mu_1)$から$(G_2,\mu_2)$への準同型写像 homomorphism である}という.
\end{Def}
\begin{Def}
群$(G_1,\mu_1)$から群$(G_2,\mu_2)$への準同型写像$f$が全単射であるとき,
{\bf $f$は$(G_1,\mu_1)$から$(G_2,\mu_2)$への同型写像 isomorphism である}という
\end{Def}
\begin{Def}
群$(G,\mu)$において,
任意の$a,b\in G$に対して$\mu(a,b)=\mu(b,a)$が成り立つとき,
{\bf 群$(G,\mu)$は可換群\index{かかんぐん@可換群} commutative group である}という\footnote{アーベル群 abelian groupともいう}.
\end{Def}
\begin{comment}
\begin{Def}
位相空間$X$と自然数$n$に対して次の手続きで決定されるアーベル群$H_n(X)$を{\bf$n$次 ホモロジー群}と呼ぶ

...

\end{Def}
\end{comment}

\section{環}
\begin{Def}
集合$R$と$R$上の二項演算$+,*$が次を満たすとき, {\bf $(R,+,*)$は環 ring である}といい, $+$を{\bf 加法}, $*$を{\bf 乗法}という. 
\begin{enumerate}
\item $(R,+)$が可換群である.
\item $(R,*)$が半群である.
\item 任意の$a,b,c\in R$に対して
\begin{align*}
a*(b+c)&=a*b+a*c\\
(a+b)*c&=a*c+b*c
\end{align*}
が成り立つ.
\end{enumerate}
\end{Def}
\begin{caution}
定義2.11の第3の条件のことを乗法の加法に対する{\bf 分配律 distributive property}という.
\end{caution}
\section{体}
\begin{Def}
環$(K,+,*)$において,
群$(K,+)$の単位元を{\bf $(K,+,*)$の零元}という.
\end{Def}
\begin{Notation}
環$(K,+,*)$における零元を$0_K$で表す.
\end{Notation}
\begin{Def}
環$(K,+,*)$において,
群$(K,*)$の単位元を{\bf $(K,+,*)$の乗法単位元}という.
\end{Def}
\begin{Notation}
環$(K,+,*)$における乗法単位元を$1_K$で表す.
\end{Notation}

\begin{Def}
環$(K,+,*)$が次を満たすとき, {\bf $(K,+,*)$は体 fieldである}という.
\begin{enumerate}
\item 乗法について零元以外の元が可換群をなす, すなわち$(K\setminus\{0_K\},*)$が可換群である.
\item 零元と乗法単位元が異なる, すなわち$0_K\neq 1_K$である.
\end{enumerate}
\end{Def}