\section{対応}
\begin{Def}
{\bf 対応}という.
\end{Def}
\begin{Notation}
集合$A$から$B$の対応$\phi$を
$\phi:A \multimap B$で表す.
\end{Notation}
\section{二項関係}
\begin{Def}
集合$A,B$の直積集合の部分集合を, 
{\bf 集合$A,B$の
二項関係 binary relation}\index{にこうかんけい@二項関係}
という.
\end{Def}
\begin{Def}
集合$A$の元$a$と集合$B$の元$b$からなる順序対$(a,b)$が
集合$A,B$の二項関係$\mathrm{R}$の元であるとき,
{\bf $a\in A$と$b\in B$の間に二項関係$\mathrm{R}$がある}という.
\end{Def}
\begin{Notation}
$a\in A$と$b\in B$の間に二項関係$\mathrm{R}\subset A\times B$があることを$a\mathrm{R}b$で表す.
\end{Notation}
\begin{Def}\label{Def:composition}
集合$A,B$の二項関係$F$と
集合$B,C$に二項関係$G$に対する
\[
\{(a,c)\in A\times C\mid \exists b\in B, ((a,b)\in F \land (b,c)\in G)\}
\]
を
{\bf 集合$A,B$の二項関係$F$と
集合$B,C$の二項関係$G$の
合成 composition\index{ごうせい@合成}
}という.
\end{Def}
\begin{Def}
二項関係の合成を生成する演算を{\bf 二項関係の合成演算}という.
\end{Def}
\begin{Notation}
二項関係の合成演算を$\circ$で表す.
\end{Notation}
\begin{Notation}
集合$A,B$の二項関係$F$と
集合$B,C$の二項関係$G$の
合成を$G\circ F$で表す.
\end{Notation}
\begin{thm}
二項関係の合成演算について結合法則が成り立つ.
\end{thm}
\begin{comment}
\begin{Prop}\label{Prop:binary relation composition}
二項関係$F\subset A\times B ,G\subset B\times C, H\subset C\times D$について
\[
(H\circ G)\circ F=H\circ (G\circ F)
\]
が成り立つ.
すなわち, 任意の$(a,d)\in A\times D$について
\begin{align*}
(a,d)\in (H\circ G)\circ F \Leftrightarrow (a,d)\in H\circ(G\circ F)
\end{align*}
が成り立つ.
\end{Prop}
\end{comment}
\subsection{二項関係の実例:順序関係}
\begin{Def}
集合$X$とそれ自身$X$の二項関係$\mathrm{R}$に関して
$
\forall x\in X,(x\mathrm{R}x)
$
という命題が常に真となるとき,
成り立つ法則を{\bf 反射法則}という.
\end{Def}
\begin{Def}
集合$X$とそれ自身$X$の二項関係$\mathrm{R}$に関して
$
\forall x\in X,\forall y\in X, ((x\mathrm{R}y\land y\mathrm{R}x)\Rightarrow x=y)
$
という命題が常に真となるとき, 成り立つ法則を{\bf 反対称法則}という.
\end{Def}
\begin{Def}
集合$X$とそれ自身$X$の二項関係$\mathrm{R}$に関して
$
\forall x\in X,\forall y\in X,\forall z\in X, ((x\mathrm{R}y\land y\mathrm{R}z)\Rightarrow x\mathrm{R}z)
$
という命題が常に真となるとき, 成り立つ法則を{\bf 推移法則}という.
\end{Def}
\begin{Def}
集合$X$とそれ自身$X$の二項関係$\preceq$が, 反射法則, 反対称法則, 推移法則をすべて満たすとき, $\preceq$を{\bf 集合$X$上の順序関係 order\index{じゅんじょかんけい@順序関係}}
という.
\begin{comment}
\begin{enumerate}
\item 任意の$x\in X$について$(x,x)\in \mathrm{R}$である.
\item 任意の$x\in X,y\in X$について$(x,y)\in \mathrm{R}$かつ$(y,x)\in \mathrm{R}$ならば$x=y$である.
\item 任意の$x\in X,y\in X,z\in X$について$(x,y)\in \mathrm{R}$かつ $(y,z)\in \mathrm{R}$ならば
$(x,z)\in \mathrm{R}$である.
\end{enumerate}
\end{comment}
\end{Def}
\begin{comment}
\begin{Def}
集合$X$上の順序関係が存在するとき,
{\bf 集合$X$は順序集合 orderd set\index{じゅんじょしゅうごう@順序集合}である}
という.
\end{Def}
\begin{Def}
{\bf 順序集合の元$x,y$の間に順序関係がある}という.
\end{Def}
\begin{Notation}
順序集合の元$x,y$の間に順序関係があるとき, $x\preceq y$で表す.
\end{Notation}
\end{comment}

\section{写像}
\begin{comment}
\begin{mean}
{\bf 規則}とは, 手続きが, それに基づいて行われるように定めた事柄のことである.\end{mean}
\begin{mean}
{\bf 決定する}とは, はっきりと決まることである.
\end{mean}
\begin{usage}
集合$A$と集合$B$に関して, ある規則が存在して, 任意の$a\in A$に対して$b\in B$が決定するとき,
{\bf 「$a\in A$と$b\in B$が対応する」},
{\bf 「$b\in B$が$a\in A$に対応する」},
{\bf 「$a\in A$に対応する$b\in B$が存在する」}といえる.
\end{usage}
\end{comment}
\begin{Def}
集合$A,B$の二項関係$f$について,
$\forall a\in A, (\exists! b\in B, ((a,b)\in f))$が常に真となるとき, $f$を{\bf 集合$A$から集合$B$への写像map\index{しゃぞう@写像}}という.
\end{Def}
\begin{Notation}
集合$A$から集合$B$への写像$f$を$f:A\rightarrow B$で表す.
\end{Notation}
\begin{Notation}
写像$f:A\rightarrow B$に関して, $a\in A$に対して$b\in B$が存在して, $(a,b)\in f$であることを$b=f(a)$で表す.
\end{Notation}
\begin{Def}
写像$f:A\rightarrow B$が,
任意の$a_1,a_2\in A$について
\[
a_1\neq a_2\Rightarrow f(a_1)\neq f(a_2)
\]
を満たすとき
{\bf 写像$f:A\rightarrow B$は
単射 injection\index{たんしゃ@単射}
である}という.
\end{Def}
\begin{Def}
写像$f:A\rightarrow B$に関して,
任意の$b\in B$に対して
とある$a\in A$が存在して
\[
b=f(a)
\]
であるとき
{\bf 写像$f:A\rightarrow B$は
全射surjection\index{ぜんしゃ@全射} 
である}という.
\end{Def}
\begin{Def}
写像$f:A\rightarrow B$が, 単射であり, かつ全射であるとき
{\bf 写像$f:A\rightarrow B$は
全単射 bijection\index{ぜんたんしゃ@全単射}
である}という.
\end{Def}

\begin{Def}
写像$f:A\rightarrow B$に対して,写像$g:B\rightarrow A$が存在して, 任意の$a\in A$に対して
\[
f(a)=b \Leftrightarrow g(b)=a
\]
が成り立つとき, 
{\bf 写像$g:B\rightarrow A$は写像$f:A\rightarrow B$の
逆写像 inverse mapping\index{ぎゃくしゃぞう@逆写像}
である}という.
\end{Def}
\begin{Notation}
写像$f$の逆写像を$f^{-1}$で表す.
\end{Notation}

\subsection{写像の演算:合成}
\begin{caution}\label{Prop:composition mapping}
写像は二項関係であるため,
合成および合成演算を定義することができる.
\end{caution}
\begin{caution}
写像の合成が写像となるように, 合成演算を定義することができる.
\end{caution}
\begin{Def}
$g\circ f:A\rightarrow C$を
\begin{align*}
(g\circ f)(a)=g(f(a))
\end{align*}
で定義する.
\end{Def}
\begin{Def}
写像$f:A\rightarrow B$と$g:B\rightarrow C$を合成して得られる写像を
{\bf 写像$f:A\rightarrow B$と$g:B\rightarrow C$の
合成写像 composition mapping\index{ごうせいしゃぞう@合成写像}}
という.
\end{Def}
\begin{Notation}
写像$f:A\rightarrow B$と$g:B\rightarrow C$の合成写像を$g\circ f$で表す.
\end{Notation}
\begin{Prop}\label{Prop:composition mapping associative law}
写像の合成演算について, 結合法則が成り立つ.

写像$f:A\rightarrow B, g:B\rightarrow C,
h:C\rightarrow D$について
\[
(h\circ g)\circ f=h\circ(g\circ f)
\]
が成り立つ.
\end{Prop}


\section{命題の証明}
\subsubsection{命題\ref{Prop:binary relation composition}の証明}
\begin{proof}
\end{proof}

