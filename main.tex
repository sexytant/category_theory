\documentclass[dvipdfmx]{jsbook}

% Language setting
% Replace `english' with e.g. `spanish' to change the document language

% Set page size and margins
% Replace `letterpaper' with `a4paper' for UK/EU standard size
\usepackage[letterpaper,top=2cm,bottom=2cm,left=3cm,right=3cm,marginparwidth=1.75cm]{geometry}

% Useful packages
\usepackage{amsmath}
\usepackage{amssymb}
\usepackage{graphicx}
\usepackage[colorlinks=true, allcolors=blue]{hyperref}
\usepackage{tikz-cd}
\usepackage{amsthm}
\usepackage{mathrsfs}
\usepackage{comment}
\usepackage{hyperref}
\usepackage{pxjahyper}
\theoremstyle{plain}

\newtheorem{thm}{定理}[chapter]
\newtheorem{Def}[thm]{定義}
\newtheorem{Notation}[thm]{記法}
\newtheorem{Prop}[thm]{命題}
\newtheorem{caution}[thm]{注意}
\newtheorem{example}[thm]{実例}
\renewcommand{\proofname}{\textbf{証明}}
\usepackage{makeidx}
\usepackage{comment}
\usepackage{listings}
\lstset{
	%プログラム言語(複数の言語に対応,C,C++も可)
 	language = haskell,
 	%背景色と透過度
 	backgroundcolor={\color[gray]{.90}},
 	%枠外に行った時の自動改行
 	breaklines = true,
 	%自動開業後のインデント量(デフォルトでは20[pt])	
 	breakindent = 10pt,
 	%標準の書体
 	basicstyle = \ttfamily\normalsize,
 	%basicstyle = {\small}
 	%コメントの書体
 	commentstyle = {\itshape \color[cmyk]{1,0.4,1,0}},
 	%関数名等の色の設定
 	classoffset = 0,
 	%キーワード(int, ifなど)の書体
 	keywordstyle = {\bfseries \color[cmyk]{0,1,0,0}},
 	%""で囲まれたなどの"文字"の書体
 	stringstyle = {\ttfamily \color[rgb]{0,0,1}},
 	%枠 "t"は上に線を記載, "T"は上に二重線を記載
	%他オプション:leftline,topline,bottomline,lines,single,shadowbox
 	frame = TBrl,
 	%frameまでの間隔(行番号とプログラムの間)
 	framesep = 5pt,
 	%行番号の位置
 	numbers = left,
	%行番号の間隔
 	stepnumber = 1,
	%右マージン
 	%xrightmargin=0zw,
 	%左マージン
	%xleftmargin=3zw,
	%行番号の書体
 	numberstyle = \tiny,
	%タブの大きさ
 	tabsize = 4,
 	%キャプションの場所("tb"ならば上下両方に記載)
 	captionpos = t
}

\makeindex
\title{圏論原論I}
\author{Sexytant}
\begin{document}
\setcounter{tocdepth}{2}
\maketitle
\tableofcontents
\newpage
学習の目安
\begin{center}
\begin{tabular}{cl}
1&準備 「集合」と「二項関係・部分関数・写像(関数)」\\
2&準備 「群・準同型写像・同型写像」と「体」\\
3&圏 「圏」\\
4&圏 「Haskellにおける圏」\\
5&関手 「関手」と「Haskellの型構築子と関手」\\
6&関手 「2変数の関手」と「型クラスとHaskの部分圏」\\
7&自然変換 「自然変換」\\
8&自然変換 「Haskにおける自然変換\\
9&定数関手 「定数関手」\\
10&定数関手 「Haskにおける定数関手」\\
11&関手圏\\
12&圏同値\\
\end{tabular}
\end{center}
\newpage
\chapter{準備}
\section{集合}
{\bf 集合 set}とは, ひとまず素朴に「ものの集まり」と定義される. 集合を構成する「もの」を{\bf 元 element}という.
\begin{Notation}
集合は各元をカンマで区切り$\{\}$で囲むことで表される. 
\end{Notation}
\begin{Notation}
とある条件を満たす$x$全体からなる集合は$\{x\mid x\text{についての条件}\}$を用いて表す.
\end{Notation}
\begin{example}$\{1,2,3\}$や$\{a,b,c\}$, 自然数全体$\mathbb{N}$, 整数全体$\mathbb{Z}$, 実数全体$\mathbb{R}$, 複素数全体$\mathbb{C}$は集合である.
\end{example}
\begin{example}
$\{2a\mid a\in\mathbb{N}\}$や$\{a^3\mid a\in\mathbb{N}\}$は集合である.
\end{example}
\begin{Notation}
$a$が集合$S$の元であることを$a\in S$で表す.
\end{Notation}
\begin{Def}
集合$A$の元のすべてが集合$B$の元であるとき{\bf $A$は$B$の部分集合 subsetである}という.  
\end{Def}
\begin{Notation}
集合$A$が集合$B$の部分集合であることを$A\subset B$で表す.
\end{Notation}
\begin{Def}
集合$S$が与えられたとき, $S$の部分集合の全体
$\mathscr{P}S=\{A\mid A\subset S\}$
を{\bf $S$の冪集合 power set}という.
\end{Def}

\begin{Def}
集合$A,B$に対して$\{(a,b)\mid a\in A, b\in B\}$を{\bf $A$と$B$の
直積 direct product}という.
\end{Def}
\begin{Notation}
$A$と$B$の
直積を$A\times B$で表す.
\end{Notation}

実は, 集合を「ものの集まり」と素朴に定義することは, 矛盾を孕んでいる. この矛盾を避けるための議論はのちに行う. 
また, 集合論の公理には立ち入らないこととする.

\section{二項関係・写像(関数)}
\begin{Def}
集合$A,B$に対して$R\subset A\times B$であるとき, {\bf $R$は$A$と$B$の二項関係 binary relation}\index{にこうかんけい@二項関係}であるという.
\end{Def}
\begin{Def}
$R$が集合$A,B$の二項関係であるとする. 
とある$a\in A, b\in B$の組$(a,b)$が$R$の元であるとき,
{\bf $a$と$b$に間に$R$の関係\index{かんけい@関係}がある}という.
\end{Def}
\begin{Notation}
$a$と$b$の間に$R$の関係があることを$aRb$と表す.
\end{Notation}
\begin{Def}
$R$が集合$A,B$の二項関係であるとする.
任意の$a\in A$について, とある$b\in B$が一意に存在して$aRb$となるとき,
{\bf $R$は$A$から$B$への写像map\index{しゃぞう@写像}である}という.
\end{Def}
\begin{Notation}
集合$A$から集合$B$への写像$f$を$f:A\rightarrow B$で表す.
\end{Notation}
\begin{caution}
以下では, 「{\bf 関数 function\index{かんすう@関数}}」と「写像」を同じ意味で用いる.
\end{caution}
\begin{Notation}
集合$A$から集合$B$への写像に$f$に関して, $b\in B$が$a\in A$に対応することを
\[
b=f(a)
\]
で表す.
\end{Notation}
\begin{Def}
集合$A$から集合$B$への写像$f$が,
任意の$a_1,a_2\in A$について
\[
a_1\neq a_2\Rightarrow f(a_1)\neq f(a_2)
\]
を満たすとき{\bf $f$は単射 injection である}という.
\end{Def}
\begin{Def}
集合$A$から集合$B$への写像$f$が,
任意の$b\in B$に対して
とある$a\in A$が存在して
\[
b=f(a)
\]
であるとき{\bf $f$は全射surjection である}という.
\end{Def}
\begin{Def}
集合$A$から集合$B$への写像$f$が, 単射であり, かつ全射であるとき{\bf $f$は全単射 bijection である}という.
\end{Def}
\begin{comment}
\begin{Def}
$R$が集合$A,B$の二項関係であるとする.

任意の$a\in A$について$b,b'\in B$が存在し,
\[
aRb\land aRb'\Rightarrow b=b'
\]
が成り立つとき, {\bf $R$は$A$から$B$への部分関数\index{ぶぶんかんすう@部分関数}}という.\footnote{ここいらない気がする}
\end{Def}
\begin{Prop}
関数は部分関数である. これは定義より明らかである.
\end{Prop}
\begin{Prop}
部分関数は二項関係である. これは定義より明らかである.
\end{Prop}
\end{comment}
\subsection{合成と結合律}
\begin{Def}
集合$A,B,C$に関する二項関係$F\in A\times B$と$G\in B\times C$について, その{\bf 合成 composition} $G\circ F\in A\times C$を
\[
G\circ F=\{(a,c)\in A\times C|\text{とある}b\in B\text{が存在して}(a,b)\in F \text{かつ} (b,c)\in G\text{である}\}
\]
で定義する.
\end{Def}
\begin{Prop}
二項関係の合成は{\bf 結合律 associative law}を満たす.
結合律とは, 集合$A,B,C,D$に対する3つの二項関係$F\in(A\times B),G\in(B\times C),H\in(C\times D)$について
\[
(H\circ G)\circ F=H\circ (G\circ F)
\]
が成り立つことをいう.
\end{Prop}

\begin{Def}
集合$A,B,C$に関する写像$f:A\rightarrow B$と$g:B\rightarrow C$について, その合成 $g\circ f:A\rightarrow C$を
\[
(g\circ f)(a)=g(f(a))
\]
で定義する. ここで$a$は$A$の元である.
\end{Def}
\begin{Prop}
写像の合成は結合律を満たす.
すなわち, 集合$A,B,C,D$に対する3つの写像$f:A\rightarrow B, g:B\rightarrow C,
h:C\rightarrow D$について
\[
(h\circ g)\circ f=h\circ(g\circ f)
\]
が成り立つ.
\end{Prop}
\begin{comment}
\begin{Def}
部分関数の合成
\end{Def}

\begin{Prop}
部分関数の合成は結合律を満たす.
\end{Prop}
\end{comment}

\section{「宇宙」}
{\bf 内包原理}とは「集合$S$の元$x$に対してtrue か falseを返す関数$\varphi:S\rightarrow\{\mathrm{true},\mathrm{false}\}$が与えられたとき, 新たな集合
$
\{x\in S\mid\varphi(x)=\mathrm{true}\}
$
を構成できる」という集合が満たすべき性質のことをいう. 

ここまで, 集合を「ものの集まり」と素朴に定義してきたが, 次のような集合$R$を内包原理に基づいて構成しようとすると矛盾が生じる.\footnote{ラッセルのパラドックスと呼ばれる}
\[
R=\{X\mid X\notin X\}
\]
もし$R\in R$であると仮定すると, $R$の定義により$R\notin R$となる.
他方, $R\notin R$と仮定すると, $R$の定義より$R\in R$となってしまう.

このような矛盾が発生しないようにするため, {\bf 「宇宙」universe}という概念を導入する.
\begin{Def}
以下の性質を満たす集合$U$を{\bf 「宇宙」universe}という.
\begin{enumerate}
\item $X\in Y\land Y\in U\Rightarrow X\in U$
\item $X\in U\land Y\in U\Rightarrow\{X,Y\}\in U$
\item $X\in U\Rightarrow \mathscr{P}X\in U\land \cup X\in U$\footnote{未定義の記法}
\item $\mathbb{N}\in U$
\item $f:A\rightarrow B$が全射で, $A\in U\land B\subset U\Rightarrow B \in U$
\end{enumerate}
\end{Def}
\begin{Def}
「宇宙」の元を{\bf 小集合 small set}という.
\end{Def}
\begin{Def}
$a$と$b$が小集合のとき関数$f:a\rightarrow b$を{\bf 小関数 small function}という.
\end{Def}
以上を定義することにより, ...

\begin{caution}
「宇宙」は小集合の全体である.
\end{caution}
\begin{Prop}
「宇宙」は小集合ではない
\end{Prop}
\begin{proof}
もし, 「宇宙」が「宇宙」の元であるとすると,...と矛盾する.
\end{proof}
\begin{Def}
「宇宙」の部分集合を{\bf 類 class} という.
\end{Def}
\begin{Prop}
「宇宙」は類である.
\end{Prop}
\begin{proof}
「宇宙」は「宇宙」の部分集合である.
\end{proof}
\begin{Def}
小集合でない類を{\bf 真類 proper class}という.
\end{Def}
\begin{Prop}
「宇宙」は真類である.
\end{Prop}
\begin{proof}
「宇宙」は小集合ではなく, かつ類である.
\end{proof}
\section{位相空間}
\begin{Def}
集合$X$に関して, 写像$d:X\times X\rightarrow \mathbb{R}$が以下の条件を満たすとき, {\bf $d$は$X$上の距離metricである}という.
\begin{enumerate}
\item 任意の$x,y\in X$について$d(x,y)=0\Leftrightarrow x=y$
\item 任意の$x,y\in X$について$d(x,y)=d(y,x)$
\item 任意の$x,y,z\in X$について$d(x,y)+d(y,z)\geq d(x,z)$
\end{enumerate}
\end{Def}
\begin{Def}
集合$X$と$X$上の距離$d$の組$(X,d)$を{\bf 距離空間 metric space}という
\end{Def}
\begin{Def}
集合$X$の部分集合$U\subset X$と$X$上の距離$d$について, 以下が成り立つとき, {\bf $U\subset X$は開集合である}という.

任意の$x\in U$に対して, とある実数$\epsilon > 0$が存在し, $d(x,y)<\epsilon$を満たす全ての$y\in X$が$y\in U$となる 
\end{Def}
\begin{Def}
{\bf 開集合系}
\end{Def}
\begin{Def}
{\bf 位相}
\end{Def}
\begin{Def}
{\bf 位相空間}
\end{Def}

\section{群・準同型写像・同型写像}
\begin{Def}
集合$A$に関する写像$\mu:A\times A\rightarrow A$
を{\bf $A$上の二項演算 binary operation $\mu$}という.
\end{Def}
\begin{Def}
集合$G$について, とある元$e\in G$が存在して, 任意の元$g\in G$に対して\[\mu(e,g)=\mu(g,e)=g\]が成り立つとき, $e$を{\bf $G$の単位元 identity element}という.
\end{Def}
\begin{Def}
集合$G$について, 任意の元$g\in G$に対して, とある元$h$が存在して \[\mu(g,h)=\mu(h,g)=e\]が成り立つとき, $h$を{\bf $g$の逆元 inverse element }という.
ここで$e$は$G$の単位元である.
\end{Def}
\begin{Notation}
集合$G$における元$g\in G$の逆元を$g^{-1}$で表す.
\end{Notation}
\begin{Def}
集合$G$と$G$上の二項演算$\mu$が以下の条件を満たすとき{\bf$(G,\mu)$は 群 group である}という
\begin{enumerate}
\item $\mu$が結合律を満たす. すなわち, 任意の$G$の元$g,h,k$に対して\[\mu(g,\mu(h,k))=\mu(\mu(g,h),k)\]が成り立つ.
\item $G$は単位元をもつ.
\item $G$の任意の元に対して逆元が存在する.
\end{enumerate}
\end{Def}
\begin{Def}
群$(G_1,\mu_1)$から群$(G_2,\mu_2)$への写像$f$が任意の$G_1$の元$g, g'$について \[f(\mu_1(g,g')) = \mu_2(f(g),f(g'))\] を満たすとき、{\bf $f$は$(G_1,\mu_1)$から$(G_2,\mu_2)$への準同型写像 homomorphism である}という.
\end{Def}
\begin{Def}
群$(G_1,\mu_1)$から群$(G_2,\mu_2)$への準同型写像$f$が全単射であるとき,
{\bf $f$は$(G_1,\mu_1)$から$(G_2,\mu_2)$への同型写像 isomorphism である}という
\end{Def}
\begin{Def}
群$(G,\mu)$において,
任意の$a,b\in G$に対して$\mu(a,b)=\mu(b,a)$が成り立つとき,
{\bf 群$(G,\mu)$は可換群\index{かかんぐん@可換群} commutative group である}という\footnote{アーベル群 abelian groupともいう}.
\end{Def}
\begin{comment}
\begin{Def}
位相空間$X$と自然数$n$に対して次の手続きで決定されるアーベル群$H_n(X)$を{\bf$n$次 ホモロジー群}と呼ぶ

...

\end{Def}
\end{comment}


\section{体}
\begin{Def}
{\bf 加法}
\end{Def}
\begin{Def}
{\bf 乗法}
\end{Def}
\begin{Def}
{\bf 分配 distributive property}
\end{Def}
\begin{Def}
以下の条件を満たす集合を{\bf 体}という.
\begin{enumerate}
\item
加法について可換群になっている.すなわち,加法について閉じていて,単位元と逆元が存在する.
\item 
乗法について可換群になっている. すなわち,乗法について閉じていて,単位元と逆元が存在する.
\item 
加法と乗法について分配法則が成り立つ.
\end{enumerate}
\end{Def}
体ではいわゆる四則演算が可能である
\section{Haskellの基礎}
関数型プログラミング言語Haskellでは圏論的な視点からライブラリが構築されている.
\subsection{型}
\subsubsection{定義済みの型}
Haskellには標準ライブラリに\verb|Int|, \verb|Integer|, 
\verb|Char|,
\verb|Float|,
\verb|Double|,
\verb|Bool|
などの型が定義済みである.
\subsubsection{新型定義}
定義済みの型をもとにタプル,リストあるいは\verb|Maybe|のような型構築子によって新たな型を無限に作り出すことができる.
\begin{example}
\verb|[Integer]|, \verb|Maybe Int|,
\verb|(Int,[Char])| などはすべてHaskellの型である.
\end{example}
\subsubsection{データ型}
\begin{lstlisting}
data Color = Red | Green | Blue 
\end{lstlisting}
データ型は型クラス\verb|Show|を付与することにより出力が可能となる.
\begin{lstlisting}
data Color = Red | Green | Blue deriving Show
\end{lstlisting}
\subsubsection{型クラス}
\begin{lstlisting}
class Foo a where
    foo :: a -> String
instance Foo Bool where
    foo True = "Bool: True"
    foo False = "Bool: False"
instance Foo Int where
    foo x = "Int: " ++ show x
instance Foo Char where
    foo x = "Char: " ++ [x]

main = do
    putStrLn $ foo True		-- Bool: True
    putStrLn $ foo (123::Int)	-- Int: 123
    putStrLn $ foo 'A'		-- Char: A
\end{lstlisting}
Foo 型クラスは任意の型(a)を受け取り、Stringを返却するメソッド foo を持っている. instance を用いてそれぞれの型が引数に指定された場合の処理を実装している.
\section{補遺}
\subsection{順序集合}
\begin{Def}
次が成り立つ演算$\curlyeqprec$を{\bf 順序}という.
\begin{enumerate}
\item $x \curlyeqprec x$
\item $x\curlyeqprec y,x\curlyeqprec y\Rightarrow x=y$
\item $x \curlyeqprec y, y \curlyeqprec z\Rightarrow x \curlyeqprec z$
\end{enumerate}
\end{Def}
\begin{Def}
元に関して順序が与えられている集合を{\bf 順序集合}という.
\end{Def}
\subsection{ホモトピー}
\begin{Def}
位相空間$X,Y$における
写像$f:X\rightarrow Y$について,
$Y$における任意の開集合$V\subset Y$に対して
\begin{align*}
f^{-1}(V)=\{x\in X\mid f(x)\in V\}
\end{align*}
が$X$の開集合となるとき,
$f$は{\bf 連続写像}であるという.
\end{Def}
\begin{Def}
位相空間$A,B$と2つの連続写像$f:A\rightarrow B,g:B\rightarrow A$に対して, 連続関数
\[
\alpha : A\times [0,1]\rightarrow B
\]
が存在し,
\begin{align*}
\alpha(x,0)=f(x)\\
\alpha(x,1)=g(x)
\end{align*}
となっているとき
{\bf $\alpha$は$f$と$g$を結ぶホモトピーである}という.
\end{Def}

\chapter{圏}
\section{圏}
\begin{Def}
{\bf 対象}の集合$\mathrm{Obj}$と
{\bf 射}の集合$\mathrm{Mor}$からなり, 以下の演算が定義されているものを{\bf 圏}という.
\begin{enumerate}
\item 射$f\in\mathrm{Mor}$には{\bf 始域}$\mathrm{dom}f$および{\bf 終域}$\mathrm{cod}f$となる対象がそれぞれ一意に定まる\footnote{$\mathrm{dom}f=A\in\mathrm{Obj}(\mathscr{C}),\mathrm{cod}f=B\in\mathrm{Obj}(\mathscr{C})$のとき$f:A\rightarrow B$とかく.}
\item $\mathrm{dom} g=\mathrm{cod}f$を満たす射$f,g\in\mathrm{Mor}$に対して,
{\bf 合成射}$g\circ f:\mathrm{dom}f\rightarrow\mathrm{cod}g$が一意に定まる
\item 射の合成は結合律を満たす. すなわち, 対象$A,B,C,D\in\rm{Obj}$についての射の列$f,g,h$が与えられたとき
\[
h\circ(g\circ f)=(h\circ g)\circ f
\]
が成り立つ.
\item 任意の対象$A\in\mathrm{Obj}$について,
次の条件を満たす{\bf 恒等射}$1_{A}:A\rightarrow A$が存在する\footnote{恒等写$1_{A}$は一意に定まるので, 定義に一意性を加えても問題ない.}

条件:任意の射の組$f:A\rightarrow B, g:B\rightarrow A$に対して
$f\circ 1_A=f$かつ$1_A\circ g=g$
\end{enumerate}
\end{Def}
\begin{Def}
同型射と逆射
\end{Def}
\begin{Notation}
圏$\mathscr{C}$の対象$A,B$に対して$f:A\rightarrow B$となる射の全体を$\mathrm{Hom}_{\mathscr{C}}(A,B)$で表す.
\end{Notation}
\begin{Prop}
ある射の逆射は存在すれば, 一意に定まる
\end{Prop}
\subsection{情報隠蔽された対象の探究に圏論が提供する方法論}
オブジェクト指向プログラミングでは,「知らせる必要のない情報は隠蔽しておくほうが安全である」という{\bf 情報隠蔽}の考え方が重要視される.
これに対して, 圏論は, 対象がもつ情報が隠蔽されている状況下で, 射のみから対象について探究するという方法論を提供する.
\begin{example}
どのような要素をもつかわからない集合$A$について写像$f:A\rightarrow A$が定義されていて$f\circ f\circ f$が恒等写像になるとする.
このとき$A$が3つの要素$a_1,a_2,a_3$をもつと仮定することができ,
\[
f(a_1)=a_2, f(a_2)=a_3, 
f(a_3)=a_1
\]
というように, これらの要素が写像$f$によって回転していると考えることができる.
\end{example}
\subsection{小圏・局所小圏・大圏}
\begin{Def}
対象の集合,射の集合がともに小集合である圏を{\bf 小圏}\index{しょうけん@小圏}という.
\end{Def}
\begin{example}
順序集合は小圏である.
\end{example}
\begin{Def}
すべての対象の組$A,B$に対して$\mathrm{Hom}_{\mathscr{C}}(A,B)$が小集合である圏$\mathscr{C}$を{\bf 局所小圏}という.
\end{Def}
\begin{Def}
小圏でない圏を{\bf 大圏}という.
\end{Def}

\begin{example}
すべての小集合を対象とし, それらの間の写像を射とする圏$\mathrm{Set}$は大圏である.
\end{example}
\begin{example}
すべての群を対象とし, それらの間の準同型写像を射とする圏$\mathrm{Grp}$は大圏である.
\end{example}
\begin{example}
すべてのアーベル群を対象とし, それらの間の準同型写像を射とする圏$\mathrm{Ab}$は大圏である.
\end{example}
\begin{example}
すべての位相空間を対象とし, それらの間の連続写像\footnote{未定義?}を射とする圏$\mathrm{Top}$は大圏である.
\end{example}
\begin{example}
ある体$k$に対して,
すべての$k$次線形空間\footnote{未定義?}を対象とし,
それらの間の$k$次線形写像\footnote{未定義?}を射とする
圏$\mathrm{Vect}_k$は大圏である.
\end{example}
\subsection{部分圏}
\begin{Def}
圏$\mathscr{A}$が圏$\mathscr{B}$に対して, 以下の$3$条件を満たすとき, {\bf $\mathscr{A}$は$\mathscr{B}$の部分圏である}という.
\begin{enumerate}
\item $\mathrm{Obj}(\mathscr{A})$が$\mathrm{Obj}(\mathscr{B})$の部分集合である.
\item
\item
\end{enumerate}
\end{Def}
\begin{example}
圏$\mathrm{Ab}$\footnote{ref}は圏$\mathrm{Grp}$\footnote{ref}の部分圏である.
\end{example}
\subsection{双対}
\begin{Def}
任意の圏$\mathscr{C}$に対して, 対象が$\mathscr{C}$と同じで, 射の向きが$\mathcal{C}$と反対になっている圏を\bf{双対圏}\index{そうついけん@双対圏}という.
\end{Def}
\begin{Notation}
圏$\mathscr{C}$の双対圏を$\mathscr{C}^{\mathrm{op}}$と表す.
\end{Notation}
\begin{caution}
双対の原理
\end{caution}
\subsection{圏の生成}
自明な圏

非自明な圏


\begin{Prop}
集合$A$に対して写像$f:A\rightarrow A$を定める.
このとき, 対象の集合$\mathrm{Obj}=\{A\}$と, 射の集合$\mathrm{Mor}=\{1_A,f\}$からなる圏を得ることができる.
ここで, 射$1_A$は$A$についての恒等射である.
\end{Prop}
\begin{proof}
$\mathrm{dom} f=\mathrm{cod}f$であるから, 合成射$f\circ f$を定めることができる.
このとき$f\circ f=1_A$もしくは$f\circ f=f$である.

$f\circ f=1_A$ならば...

...

となる.
一方, $f\circ f=1_A$ならば...

...

となる.
以上より, いずれの場合も, 射の合成が結合律を満たすことがわかる.
したがって$f\circ f=1_A$と定めても$f\circ f=f$と定めても圏を生成することができる.
\end{proof}
\begin{Prop}
集合$A$に対して写像$f:A\rightarrow A$を定め, 正の整数$n$に対して$f^n=\underbrace{f\circ f \circ \dots \circ f}_{n}$
とする.
このとき, 対象の集合$\mathrm{Obj}=\{A\}$と, 射の集合$\mathrm{Mor}=\{1_A,f,f^2,\dots,f^n,\dots\}$からなる圏を得ることができる.
ここで, 射$1_A$は$A$についての恒等射である.
\end{Prop}
\begin{proof}
\end{proof}
\begin{Prop}
集合$A,B$に対して,写像$f:A\rightarrow B$と$g:\rightarrow A$を定める.
このとき, 対象の集合$\mathrm{Obj}=\{A,B\}$と射の集合$\mathrm{Mor}=\{1_A,1_B,f,g,f\circ g\}$からなる圏を得ることができる.
ここで, 射$1_A,1_B$はそれぞれ$A,B$についての恒等射である.
\end{Prop}
\begin{proof}
\end{proof}

{\bf 生成系}

{\bf 生成元}

{\bf 関係}

\begin{Def}
圏の積を次で定義する
\end{Def}
\begin{Prop}
圏の積は圏である.
\end{Prop}
\section{Haskellにおける圏 (Hask)}

すべてのHaskellの型を対象とし, それらの間の関数を射とする圏Haskは小圏である.
\begin{caution}
Haskellにおいて, 型\verb|A|,\verb|B|に対して, 型構築子によってつくられる\verb|A->B|は1つの型となる.
\end{caution}
\section{まとめ}
\chapter{関手}
\section{関手}
\begin{Def}
圏$\mathscr{A}$の対象$\mathrm{Obj}(\mathscr{A})$から圏$\mathscr{B}$の対象$\mathrm{Obj}(\mathscr{B})$への関数を
{\bf $\mathscr{A}$から$\mathscr{B}$への対象関数}という.
\end{Def}
\begin{Def}
{\bf 射関数}
\end{Def}
\begin{Def}
圏$\mathscr{A},\mathscr{B}$に対する対称関数と射関数の組を{\bf $\mathscr{A}$から$\mathscr{B}$への関手}という.
\end{Def}
\begin{example}
順序を保存する写像
\end{example}
\begin{example}
$n$次ホモロジー関手
\end{example}
\begin{example}
\end{example}
\begin{example}
\end{example}
\subsection{反変関手}
\begin{Def}
圏$A^{\mathrm{op}}$から圏$B$への関手
を{\bf 圏$A$から圏$B$への反変関手}という.
\end{Def}
\begin{example}
\end{example}
\begin{Notation}
\end{Notation}
\begin{caution}
\end{caution}
\subsection{定数関手}
\begin{Def}
圏$\mathscr{A}$の任意の対象$A$を圏$\mathscr{B}$のただ一つの対象$B_0$に写し,
$\mathscr{A}$の任意の射$f$を圏$\mathscr{B}$の恒等射に写す関手を
{\bf 定数関手}\index{ていすうかんしゅ@定数関手}という
\end{Def}

\subsection{忠実関手と充満関手}
\begin{Def}
圏$\mathscr{A}$
から圏$\mathscr{B}$への関手$F$に関して,
集合$\{(A_1,A_2)\mid A_1,A_2\in\mathrm{Obj}(\mathscr{A})\}$
から
集合$\{(F(A_1),F(A_2))\mid\mathrm{Obj}(\mathscr{A}))\}$
への写像が単射となっているとき,
{\bf 関手$F$は忠実}であるという.
\end{Def}
\begin{Def}
圏$\mathscr{A}$から圏$\mathscr{B}$への関手$F$に関して,
集合$\{(A_1,A_2)\mid A_1,A_2\in\mathrm{Obj}(\mathscr{A})\}$
から
集合$\{(F(A_1),F(A_2))\mid\mathrm{Obj}(\mathscr{A}))\}$
への写像が全射となっているとき,
{\bf 関手$F$は充満}であるという.
\end{Def}
\begin{Def}圏$\mathscr{A}$から圏$\mathscr{B}$への関手$F$が忠実かつ充満であるとき
{\bf 関手$F$は充満忠実である}という
\end{Def}
\begin{Def}
圏$\mathscr{A}$が圏$\mathscr{A}$の部分圏であり, 関手$F:\mathscr{A}\rightarrow\mathscr{B}$が充満であるとき,
{\bf 圏$\mathscr{A}$は圏 $\mathscr{B}$の充満部分圏である}という.
\end{Def}
\begin{example}
...充満忠実である.
\end{example}
\begin{example}
...忠実だが充満でない
\end{example}
\begin{example}
充満だが忠実でない
\end{example}
\begin{example}
複素数...

...

...忠実だが充満でない. (例1.30)
\end{example}
\subsection{埋め込み関手}
\subsection{忘却関手}
\section{Haskellの型構築子と関手}
\subsubsection{List関手}
Haskellにおける型構築子\verb|[]|は任意の型\verb|A|に対して型\verb|[A]|を対応させる.
これは, HaskからHaskへの対称関数とみなせる.
型\verb|A|と型\verb|B|および関数\verb|f::A->B|が与えられとき\verb|map f::[A]->[B]|が決定される.

...

型構築子\verb|[]|は{\bf List関手}\index{りすとかんしゅ@List関手}と呼ばれる
\subsubsection{Maybe関手}
Haskellにおける型構築子\verb|Maybe|は...

...

型構築子\verb|Maybe|は{\bf Maybe関手}\index{めいびーかんしゅ@Maybe関手}
と呼ばれる.
\subsubsection{Tree関手}
一般に木構造を生成する型構築子は関手にできる. これを{\bf Tree関手}と呼ぶ.

\lstinputlisting{src/haskell/Tree.hs}
\section{2変数の関手}
{\bf Hom関手}
\lstinputlisting{src/haskell/homfunctors.hs}
\section{型クラスとHaskの部分圏}
{\bf ソート関手}
\lstinputlisting{src/haskell/sort.hs}
\section{まとめ}
\chapter{自然変換}
\section{自然変換}
\begin{Def}
圏$\mathscr{A}$から圏$\mathscr{B}$への関手$F,G$について,
\begin{enumerate}
\item $\mathscr{A}$の任意の対象$A$ 
に対して
$\mathscr{B}$の射$\alpha_A:FA\rightarrow GA$
が存在し
\item $\mathscr{A}$の任意の射$f:A_1\rightarrow A_2$
が
\[
\alpha_{A_2}\circ Ff
=Gf\circ\alpha_{A_1}
\]
を満たすとき
\end{enumerate}に対して
{\bf 集合$\alpha=\{\alpha_A\mid A\in\mathrm{Obj}(\mathscr{A})\}$は関手$F$から関手$G$への自然変換\index{しぜんへんかん@自然変換}である}という.
\end{Def}
\begin{Notation}
関手$F$から関手$G$への自然変換$\alpha$を$\alpha:F\rightarrow G$で表す.
\end{Notation}
\begin{Notation}
関手$F$から関手$G$への自然変換$\alpha$を次の図表で表す.
\end{Notation}
\section{Haskにおける自然変換}
\subsection{concat}
\subsection{List関手からMaybe関手への自然変換safehead}
\subsection{concatとsafeheadの垂直合成}
\subsection{二分木からリストへのflatten関数}

\section{Haskにおける定数関手}
\subsection{length関数}
\section{まとめ}
\chapter{関手圏}
\section{関手圏}
\begin{Prop}
...は圏である.
\end{Prop}
\begin{Def}
{\bf $\mathscr{A}$から$\mathscr{B}$への関手圏}という.
\end{Def}
\begin{Notation}
$\mathscr{A}$から$\mathscr{B}$への関手圏を$[\mathscr{A},\mathscr{B}]$
で表す\end{Notation}
\begin{Def}
関手圏における同型射を自然同型という.
\bf{自然同型}
\end{Def}
\begin{Prop}
自然変換...が自然同型であること,

...が同型者であることは同値である.
\end{Prop}
\chapter{圏同値}
\section{圏同値}
\begin{Def}
\bf{圏同値}
\end{Def}

\section{Haskにおける自然同型}
\subsection{mirror関数}
Tree関手からTree関手への自然同型
\subsection{Maybe関手とEither()関手の間の自然同型}
\section{まとめ}
aa
\printindex
\end{document}
