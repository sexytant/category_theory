\documentclass{jsbook}

% Language setting
% Replace `english' with e.g. `spanish' to change the document language

% Set page size and margins
% Replace `letterpaper' with `a4paper' for UK/EU standard size
\usepackage[letterpaper,top=2cm,bottom=2cm,left=3cm,right=3cm,marginparwidth=1.75cm]{geometry}

% Useful packages
\usepackage{amsmath}
\usepackage{amssymb}
\usepackage{graphicx}
\usepackage[colorlinks=true, allcolors=blue]{hyperref}
\usepackage{tikz-cd}
\usepackage{amsthm}
\usepackage{mathrsfs} 
\theoremstyle{plain}

\newtheorem{thm}{定理}[chapter]
\newtheorem{Def}{定義}[chapter]
\newtheorem{Notation}{記法}[chapter]
\newtheorem{Prop}{命題}[chapter]
\newtheorem{caution}{注意}[chapter]
\newtheorem{example}{例}[chapter]
\usepackage{makeidx}
\usepackage{comment}
\usepackage{listings}
\lstset{
	%プログラム言語(複数の言語に対応,C,C++も可)
 	language = haskell,
 	%背景色と透過度
 	backgroundcolor={\color[gray]{.90}},
 	%枠外に行った時の自動改行
 	breaklines = true,
 	%自動開業後のインデント量(デフォルトでは20[pt])	
 	breakindent = 10pt,
 	%標準の書体
 	basicstyle = \ttfamily\normalsize,
 	%basicstyle = {\small}
 	%コメントの書体
 	commentstyle = {\itshape \color[cmyk]{1,0.4,1,0}},
 	%関数名等の色の設定
 	classoffset = 0,
 	%キーワード(int, ifなど)の書体
 	keywordstyle = {\bfseries \color[cmyk]{0,1,0,0}},
 	%""で囲まれたなどの"文字"の書体
 	stringstyle = {\ttfamily \color[rgb]{0,0,1}},
 	%枠 "t"は上に線を記載, "T"は上に二重線を記載
	%他オプション:leftline,topline,bottomline,lines,single,shadowbox
 	frame = TBrl,
 	%frameまでの間隔(行番号とプログラムの間)
 	framesep = 5pt,
 	%行番号の位置
 	numbers = left,
	%行番号の間隔
 	stepnumber = 1,
	%右マージン
 	%xrightmargin=0zw,
 	%左マージン
	%xleftmargin=3zw,
	%行番号の書体
 	numberstyle = \tiny,
	%タブの大きさ
 	tabsize = 4,
 	%キャプションの場所("tb"ならば上下両方に記載)
 	captionpos = t
}

\makeindex
\title{圏論原論I}
\author{Hirokichi Tanaka}
\begin{document}
\setcounter{tocdepth}{2}
\maketitle
\tableofcontents
\newpage
学習の目安
\begin{center}
\begin{tabular}{cl}
1&準備 「集合とユニバース」と「二項関係・部分関数・写像(関数)」\\
2&準備 「群・準同型写像・同型写像」と「体」\\
3&圏 「圏」\\
4&圏 「Haskellにおける圏」\\
5&関手 「関手」と「Haskellの型構築子と関手」\\
6&関手 「2変数の関手」と「型クラスとHaskの部分圏」\\
7&自然変換 「自然変換」\\
8&自然変換 「Haskにおける自然変換\\
9&定数関手 「定数関手」\\
10&定数関手 「Haskにおける定数関手」\\
11&関手圏\\
12&圏同値\\
\end{tabular}
\end{center}
\newpage
\chapter{準備}
\section{集合}
集合とは, ひとまず素朴に「ものの集まり」と定義される. 
\begin{example}$\{1,2,3\}$や$\{a,b,c\}$, 自然数全体$\mathbb{N}$, 整数全体$\mathbb{Z}$, 実数全体$\mathbb{R}$, 複素数全体$\mathbb{C}$は集合である.
\end{example}
...を{\bf 元}という.
\begin{Notation}
$a\in S$
\end{Notation}
\begin{Def}
...を{\bf 部分集合}という.
\end{Def}
\begin{Notation}
$A\subset B$
\end{Notation}
\begin{Def}
元に関して{\bf 順序}が与えられている集合を{\bf 順序集合}という.
順序$\curlyeqprec$は次で定義される.
\begin{enumerate}
\item $x \curlyeqprec x$
\item $x\curlyeqprec y,x\curlyeqprec y\rightarrow x=y$
\item $x \curlyeqprec y, y \curlyeqprec z\rightarrow x \curlyeqprec z$
\end{enumerate}
\end{Def}
\begin{Def}
集合$S$が与えられたとき, $S$の部分集合の全体
$
\mathscr{P}S=\{a:\textrm{set}\mid a\subset S\}
$\footnote{記法が未定義}
を{\bf $S$の冪集合}という.
\end{Def}
集合を「ものの集まり」と素朴に定義することは, 矛盾を孕んでいる. この矛盾を避けるための議論はのちに行う. 

\begin{Def}
{\bf 積集合}
\end{Def}
\begin{Notation}
$A\times B$
\end{Notation}
\section{二項関係・部分関数・写像(関数)}
\begin{Def}
集合$A,B$に対して$R\subset A\times B$\footnote{積集合が未定義...}であるとき, {\bf $R$は$A$と$B$の二項関係}\index{にこうかんけい@二項関係}であるという.
\end{Def}
\begin{Def}
$R$が集合$A,B$の二項関係であるとする. 
各集合の元$a\in A$と$b\in B$について$(a,b)\in R$であるとき,
{\bf $a$と$b$に間に$R$の関係\index{かんけい@関係}がある}という.
\end{Def}
\begin{Notation}
$a$と$b$の間に$R$の関係があることを$aRb$と表す.
\end{Notation}
\begin{Def}
{\bf 全射}
\end{Def}
\begin{Def}
$R$が集合$A,B$の二項関係であるとする.
任意の$a\in A$について, とある$b\in B$が一意に存在して$aRb$となるとき,
{\bf $R$は$A$から$B$への写像\index{しゃぞう@写像}である}という.
\end{Def}
\begin{Notation}
$A$から$B$への写像である$f$を$f:A\rightarrow B$で表す.
\end{Notation}
\begin{caution}
以下では, 「{\bf 関数\index{かんすう@関数}}」と「写像」を同じ意味で用いる.
\end{caution}
\begin{Def}
$R$が集合$A,B$の二項関係であるとする.

任意の$a\in A$について$b,b'\in B$が存在し,
\[
aRb\land aRb'\Rightarrow b=b'
\]
が成り立つとき, {\bf $R$は$A$から$B$への部分関数\index{ぶぶんかんすう@部分関数}}という.
\end{Def}
\begin{Prop}
関数は部分関数である. これは定義より明らかである.
\end{Prop}
\begin{Prop}
部分関数は二項関係である. これは定義より明らかである.
\end{Prop}
\subsection{合成と結合律}
\begin{Def}
集合$A,B,C$に対する2つの写像$f:A\rightarrow B$と$g:B\rightarrow C$について, その{\bf 合成写像} $g\circ f:A\rightarrow C$を
\[
(g\circ f)(a)=g(f(a))
\]
で定義する. ここで$a$は$A$の元である.
\end{Def}
\begin{Def}
部分関数の合成
\end{Def}
\begin{Def}
二項関係の合成を次で定義する

...
\end{Def}
\begin{Prop}
二項関係の合成は結合律を満たす
\end{Prop}
\begin{caution}
一般に, 結合律とは...
\end{caution}
\begin{Prop}
写像の合成は{\bf 結合律}を満たす.
すなわち, 集合$A,B,C,D$に対する3つの写像$f:A\rightarrow B, g:B\rightarrow C,
h:C\rightarrow D$について
\[
(h\circ g)\circ f=h\circ(g\circ f)
\]
が成り立つ.
\end{Prop}
\begin{Prop}
部分関数の合成は結合律を満たす.
\end{Prop}
\section{ユニバース}
{\bf 内包原理}とは「集合$S$の元$x$に対してtrue か falseを返す関数$\varphi:S\rightarrow\{\mathrm{true},\mathrm{false}\}$が与えられたとき, 新たな集合
$
\{x\in S\mid\varphi(x)=\mathrm{true}\}
$
を構成できる」という集合が満たすべき性質のことをいう. 

ここまで, 集合を「ものの集まり」と素朴に定義してきたが, 次のような集合$R$を内包原理に基づいて構成しようとすると矛盾が生じる.\footnote{ラッセルのパラドックスと呼ばれる}
\[
R=\{x\mid x\notin x\}
\]
もし$R\in R$であると仮定すると, $R$の定義により$R\notin R$となる.
他方, $R\notin R$と仮定すると, $R$の定義より$R\in R$となってしまう.

このような矛盾が発生しないようにするため, ユニバースという概念を導入する.
\begin{Def}
以下の性質を満たす$U$を{\bf ユニバース}という.
\begin{enumerate}
\item $x$が$y$の元であり, かつ$y$が$U$の元ならば, $x$は$U$の元である.
\item $x$が$U$の元であり, かつ$y$が$U$の元ならば, $\{x,y\}$は$U$の元である.
\item $x$が$U$の元ならば, 冪集合$\mathscr{P}x$が$U$の元であり, かつ$\cup x$も$U$の元である
\item $w=\{0,1,2,\dots\}$は$U$の元である.
\item $f:a\rightarrow b$が全射\footnote{未定義...}で, $a$が$U$の元であり, かつ$b$が$U$の部分集合ならば, $b$は$U$の元である
\end{enumerate}
\end{Def}
\begin{Def}
ユニバースの要素$u\in U$を{\bf 小さい集合}という.
\end{Def}
\begin{caution}
ユニバースは小さい集合の全体である.
\end{caution}
\begin{Prop}
ユニバース$U$は小さい集合ではない
\end{Prop}
\begin{Def}
$a$と$b$が小さい集合のとき関数$f:a\rightarrow b$を{\bf 小さい関数}という.
\end{Def}


\begin{Def}
ユニバースの部分集合を{\bf クラス}という.
\end{Def}
\begin{Prop}
ユニバース$U$はクラスである.
\end{Prop}
\begin{Def}
小さい集合でないクラスを{\bf 真のクラス}という.
\end{Def}
\begin{Prop}
ユニバース$U$は真のクラスである.
\end{Prop}
\section{群・準同型写像・同型写像}
\begin{Def}
\bf{群}
\end{Def}
\begin{Def}
\bf{準同型写像}
\end{Def}
\begin{Def}
\bf{同型写像}
\end{Def}
\begin{Def}
{\bf アーベル群}\index{あーべるぐん@アーベル群}
\end{Def}
\begin{Def}
$n$次{\bf ホモロジー群}
\end{Def}
\begin{Def}
{\bf 可換群}
\end{Def}

\section{体}
\begin{Def}
以下の条件を満たす集合を{\bf 体}という.
\begin{enumerate}
\item
加法について可換群になっている.すなわち,加法について閉じていて,単位元と逆元が存在する.
\item 
乗法について可換群になっている. すなわち,乗法について閉じていて,単位元と逆元が存在する.
\item 
加法と乗法について分配法則が成り立つ.
\end{enumerate}
\end{Def}
体では四則演算が可能である
\section{Haskellの基礎}
\subsection{型}
\subsubsection{定義済みの型}
Haskellには標準ライブラリに\verb|Int|, \verb|Integer|, 
\verb|Char|,
\verb|Float|,
\verb|Double|,
\verb|Bool|
などの型が定義済みである.
\subsubsection{新型定義}
定義済みの型をもとにタプル,リストあるいは\verb|Maybe|のような型構築子によって新たな型を無限に作り出すことができる.
\begin{example}
\verb|[Integer]|, \verb|Maybe Int|,
\verb|(Int,[Char])| などはすべてHaskellの型である.
\end{example}
\subsubsection{データ型}
\begin{lstlisting}
data Color = Red | Green | Blue 
\end{lstlisting}
データ型は型クラス\verb|Show|を付与することにより出力が可能となる.
\begin{lstlisting}
data Color = Red | Green | Blue deriving Show
\end{lstlisting}
\subsubsection{型クラス}
\begin{lstlisting}
class Foo a where
    foo :: a -> String
instance Foo Bool where
    foo True = "Bool: True"
    foo False = "Bool: False"
instance Foo Int where
    foo x = "Int: " ++ show x
instance Foo Char where
    foo x = "Char: " ++ [x]

main = do
    putStrLn $ foo True		-- Bool: True
    putStrLn $ foo (123::Int)	-- Int: 123
    putStrLn $ foo 'A'		-- Char: A
\end{lstlisting}
Foo 型クラスは任意の型(a)を受け取り、Stringを返却するメソッド foo を持っている. instance を用いてそれぞれの型が引数に指定された場合の処理を実装している.

\chapter{圏}
\section{圏}
\begin{Def}
{\bf 圏}$\mathscr{C}$は{\bf 対象}の集合$\mathrm{Obj}(\mathscr{C})$と
{\bf 射}の集合$\mathrm{Mor}(\mathscr{C})$からなる, 以下の演算が定義されているもののことをいう.
\begin{enumerate}
\item 射$f\in\mathrm{Mor}(\mathscr{C})$には{\bf 始域}$\mathrm{dom}f$および{\bf 終域}$\mathrm{cod}f$となる対象がそれぞれ一意に定まる\footnote{$\mathrm{dom}f=A\in\mathrm{Obj}(\mathscr{C}),\mathrm{cod}f=B\in\mathrm{Obj}(\mathscr{C})$のとき$f:A\rightarrow B$とかく.}
\item $\mathrm{dom} g=\mathrm{cod}f$を満たす射$f,g\in\mathrm{Mor}(\mathscr{C})$に対して,
{\bf 合成射}$g\circ f:\mathrm{dom}f\rightarrow\mathrm{cod}g$が一意に定まる
\item 射の合成は結合律を満たす. すなわち, 対象$A,B,C,D\in\rm{Obj}$についての射の列$f,g,h$が与えられたとき
\[
h\circ(g\circ f)=(h\circ g)\circ f
\]
が成り立つ.
\item 任意の対象$A\in\mathrm{Obj}(\mathscr{C})$について,
次の条件を満たす{\bf 恒等射}$1_{A}:A\rightarrow A$が存在する\footnote{恒等写$1_{A}$は一意に定まるので, 定義に一意性を加えても問題ない.}

条件:任意の射の組$f:A\rightarrow B, g:B\rightarrow A$に対して
$f\circ 1_A=f$かつ$1_A\circ g=g$
\end{enumerate}
\end{Def}
\begin{Def}
同型射と逆射
\end{Def}
\begin{Notation}
圏$\mathscr{C}$の対象$A,B$に対して$f:A\rightarrow B$となる射の全体を$\mathrm{Hom}_{\mathscr{C}}(A,B)$で表す.
\end{Notation}
\begin{Prop}
ある射の逆射は存在すれば, 一意に定まる
\end{Prop}
\subsection{情報隠蔽された対象の探究に圏論が提供する方法論}
オブジェクト指向プログラミングでは,「知らせる必要のない情報は隠蔽しておくほうが安全である」という{\bf 情報隠蔽}の考え方が重要視される.
これに対して, 圏論は, 対象がもつ情報が隠蔽されている状況下で, 射のみから対象について探究するという方法論を提供する.
\begin{example}
どのような要素をもつかわからない集合$A$について写像$f:A\rightarrow A$が定義されていて$f\circ f\circ f$が恒等写像になるとする.
このとき$A$が3つの要素$a_1,a_2,a_3$をもつと仮定することができ,
\[
f(a_1)=a_2, f(a_2)=a_3, 
f(a_3)=a_1
\]
というように, これらの要素が写像$f$によって回転していると考えることができる.
\end{example}
\subsection{小圏・局所小圏・大圏}
\begin{Def}
対象の集合,射の集合がともに小さい集合である圏を{\bf 小圏}\index{しょうけん@小圏}という.
\end{Def}
\begin{example}
順序集合は小圏である.
\end{example}
\begin{Def}
すべての対象の組$A,B$に対して$\mathrm{Hom}_{\mathscr{C}}(A,B)$が小さい集合である圏$\mathscr{C}$を{\bf 局所小圏}という.
\end{Def}
\begin{Def}
小圏でない圏を{\bf 大圏}という.
\end{Def}

\begin{example}
すべての小さな集合を対象とし, それらの間の写像を射とする圏$\mathrm{Set}$は大圏である.
\end{example}
\begin{example}
すべての群を対象とし, それらの間の準同型写像を射とする圏$\mathrm{Grp}$は大圏である.
\end{example}
\begin{example}
すべてのアーベル群を対象とし, それらの間の準同型写像を射とする圏$\mathrm{Ab}$は大圏である.
\end{example}
\begin{example}
すべての位相空間を対象とし, それらの間の連続写像を射とする圏$\mathrm{Top}$は大圏である.
\end{example}
\begin{example}
ある体$k$に対して,
すべての$k$次線形空間を対象とし,
それらの間の$k$次線形写像を射とする
圏$\mathrm{Vect}_k$は大圏である.
\end{example}
\subsection{部分圏}
\begin{Def}
圏$\mathscr{A}$が圏$\mathscr{B}$に対して, 以下の$3$条件を満たすとき, {\bf $\mathscr{A}$は$\mathscr{B}$の部分圏である}という.
\begin{enumerate}
\item $\mathrm{Obj}(\mathscr{A})$が$\mathrm{Obj}(\mathscr{B})$の部分集合である.
\item
\item
\end{enumerate}
\end{Def}
\begin{example}
圏$\mathrm{Ab}$は圏$\mathrm{Grp}$の部分圏である.
\end{example}
\subsection{双対}
\begin{Def}
任意の圏$\mathscr{C}$に対して, 対象が$\mathscr{C}$と同じで, 射の向きが$\mathcal{C}$と反対になっている圏を\bf{双対圏}\index{そうついけん@双対圏}という.
\end{Def}
\begin{Notation}
圏$\mathscr{C}$の双対圏を$\mathscr{C}^{\mathrm{op}}$と表す.
\end{Notation}
\begin{caution}
双対の原理
\end{caution}
\subsection{圏の生成}
\begin{Def}
生成元
\end{Def}
\begin{Def}
生成系
\end{Def}
\begin{Def}
圏の積
\end{Def}
\section{Haskellにおける圏 (Hask)}

すべてのHaskellの型を対象とし, それらの間の関数を射とする圏Haskは小圏である.
\begin{caution}
Haskellにおいて, 型\verb|A|,\verb|B|に対して, 型構築子によってつくられる\verb|A->B|は1つの型となる.
\end{caution}
\section{まとめ}
\chapter{関手}
\section{関手}
\begin{Def}
圏$\mathscr{A}$から$\mathscr{B}$への{\bf 関手}は{\bf 対象関数}$F_0:\mathrm{Obj}(\mathscr{A})\rightarrow\mathrm{Obj}(\mathscr{B})$と
{\bf 射関数}$F_1$からなるもののことをいう.
\end{Def}

\subsection{反変関手}
\begin{Def}
{\bf 反変関手}
\end{Def}

\subsection{忠実関手と充満関手}
\begin{Def}
{\bf 忠実}
\end{Def}
\begin{Def}
{\bf 充満}
\end{Def}
\begin{Def}
{\bf 充満忠実}
\end{Def}
\begin{Def}
{\bf 充満部分圏}
\end{Def}

\section{Haskellの型構築子と関手}
\subsubsection{List関手}
Haskellにおける型構築子\verb|[]|は任意の型\verb|A|に対して型\verb|[A]|を対応させる.
これは, HaskからHaskへの対称関数とみなせる.
型\verb|A|と型\verb|B|および関数\verb|f::A->B|が与えられとき\verb|map f::[A]->[B]|が決定される.

...

型構築子\verb|[]|は{\bf List関手}\index{りすとかんしゅ@List関手}と呼ばれる
\subsubsection{Maybe関手}
Haskellにおける型構築子\verb|Maybe|は...

...

型構築子\verb|Maybe|は{\bf Maybe関手}\index{めいびーかんしゅ@Maybe関手}
と呼ばれる.
\subsubsection{Tree関手}
一般に木構造を生成する型構築子は関手にできる. これを{\bf Tree関手}と呼ぶ.

\lstinputlisting{src/haskell/Tree.hs}
\section{2変数の関手}

\lstinputlisting{src/haskell/homfunctors.hs}
\section{型クラスとHaskの部分圏}
\lstinputlisting{src/haskell/sort.hs}
\section{まとめ}
\chapter{自然変換}
\section{自然変換}
\begin{Def}
{\bf 自然変換}\index{しぜんへんかん@自然変換}
\end{Def}
\section{Haskにおける自然変換}
\subsection{concat}
\subsection{safehead}
\subsection{concatとsafeheadの垂直合成}
\subsection{flatten関数}
\section{まとめ}
\chapter{定数関手}
\section{定数関手}
\begin{Def}
{\bf 定数関手}\index{ていすうかんしゅ@定数関手}
\end{Def}
\begin{Def}
\end{Def}
\section{Haskにおける定数関手}
\subsection{length関数}
\section{まとめ}
\chapter{関手圏}
\section{関手圏}
\begin{Def}
\bf{自然同型}
\end{Def}
\chapter{圏同値}
\section{圏同値}
\begin{Def}
\bf{圏同値}
\end{Def}

\section{Haskにおける自然同型}
\subsection{mirror関数}
\subsection{Maybe関手とEither()関手の間の自然同型}
\section{まとめ}
aa
\printindex
\end{document}
